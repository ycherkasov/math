\section{Формула включений-исключений.}

\subsection{Теорема}

Путсь имеется N объектов $a_1, a_2, a_3, \ldots, a_n$, и у них имеются свойства $P(1), P(2), \ldots ,P(k)$
Обозначим через $N_i$ количество объектов, обладающих свойством $P(i)$
Вообще, индексы $N_{i_1, i_2, \ldots, i_k}$ - обзначим кол-во объектов, 
обладающих свойствами $P(i_1), P(i_2), \ldots , P_{i_k}$
Тогда кол-во объектов, $N(0)$ не обладающими ни одним из свойств. можно посчитать по формуле
$$
N(0) =  N 
- \sum\limits_{i}^{} N_{i} 
+ \sum\limits_{i_1, i_2}^{} N_{i_1, i_2} 
- \sum\limits_{i_1, i_2, i_3}^{} N_{i_1, i_2, i_3} 
+ \ldots 
+ (-1)^k \sum\limits_{i_1, \ldots, i_k}^{} N_{i_1, \ldots i_k} 
+ \ldots
+ (-1)^n N_{i_1, \ldots, i_n}^{k}
$$
Т.е. из общего кол-ва объектов вычтем кол-во объектов, обладающих одним свовом, прибавить кол-во объектов, облад. 2 св-вами, прибавит кол-во объектов, вычесть кол-во объектов, 3 св-вами и т.д., 1до количества объектов, обладающих всеми свойствами (там будет уже не сумма, а просто количество объектов, обладащих всеми свойствами, если такой конечно есть)

Док-во:

Объект, не обладающий ни одним из свойств, посчитан только один раз в $N$. Объект со свойством $P(i)$ посчитан один раз в $N$ и один раз в $N_i$, следовательно, он учтен $1-1=0$ раз. В общем случае объект, обладающий $r$ свойствами: $(j_1, j_2, \dots, j_r)$, мы учли в $N_{i_1, i_2, \dots, i_k}$, причем только в том
случае, если $(i_1, i_2, \dots, i_k) \subset (j_1, j_2, \dots, j_r)$. Таким образом, в $k$-й сумме он учтен $C_r^k$ раз.

% cut from here
% % % % % % % % % % % % %
%Если есть объект, не обладающий ни одним из свойств, мы его посчитали как разность всех объектов, и объектов, обладающим хотя бы одном свойством (посчитали ровно один раз).

%Теперь посмотрим, сколько раз мы посчитали объект, обладающий ровно одним свой свойством - один раз как N и один раз как сумму объектов, обладающих одним свойством. 
%// TODO выделить цветом
%$$
%N(0) =  N - \sum\limits_{i}^{k} N_{i} +  \sum\limits_{i_1, i_2}^{k} N_{i_1, i_2} - \sum\limits_{i_1, i_2, i_3}^{k} N_{i_1, i_2, i_3} + \ldots + (-1)^k \sum\limits_{i_1, \ldots, i_k}^{k} N_{i_1, \ldots i_k} + (-1)^n N_{i_1, \ldots, i_n}^{k}
%$$


%И в общем случае - объект, обладающий r свойствами: $(j_1, j_2, ,j_r)$, мы учли его в общем числе объектов N и в подмножестве индексов $N_{i_1, i_2, , i_k}$, причем только в том случае, если $(i_1, i_2, , i_k) \subset (j_1, j_2, ,j_m)$, по определению в k-й сумме $C_t^k$.

%В k-й сумме это будет количество выбрать k-элементное подмножество из множества r элементов, т.е. 
% % % % % % % % % % %
Т.е. объект мы посчитали всего 
$$
C_{r}^{0} - C_{r}^{1} + C_{r}^{2} -
 \ldots + (-1)^k C_{r}^{k} + (-1)^r C_{r}^{r} = (1 - 1)^r = 0
$$ 
Четные с плюсом, нечетные с минусом, т.е. по $(a-b)^r$ по биному Ньютона.
чтд

\subsection{Пример. Задача о беспорядках}

Пусть имеется N упорядоченных мест мест в театре, и пришедшие N человек расселись в случайном порядке.
Требуется установить количество перестановок, в которых ни один человек не занял свое место.

Т.е., формально

$\{1,2,\ldots, n \}$ 

$\{a_1,a_2,\ldots, a_n \}$ - перестановка n чисел

Сколько есть перестановок, таких что $a_i \neq i \forall i$?

Общее кол-во перестановок n!

$P_i$ означает, что $i$-й элемент попал на свое место.

Остальные можем переставить $(n-1)!$ способами, значит, $N_i=(n-1)!$.



TODO В лекции равно $N_{i_1, i_2, , i_k} = (n-k)!$
Если фиксированы элементы $i_1$, $i_2$, \dots, $i_k$, то остается $ (n-k)!$ вариантов,
значит по формуле включения-исключения имеем

$$
n! - C_{n}^{1} (n-1)! + C_{n}^{2} (n-2)! - C_{n}^{3} (n-3)! + \ldots + (-1)^n C_{n}^{n} (0!)
$$

Обратим внимание на то, что написано на k-м месте. Это не что иное как
$$
C_n^k(n-k)!= \frac{n!(n-k)!}{k!(n-k)!} = \frac{n!}{k!}
$$
Т.е. в каждой скобке есть $n!$, который можно вынести

$$
n! (1 - 1 + 1/2! -1/3! + 1/4! - ... + (-1)^n(1/n!)) \approx \frac{n!}{e}
$$
Это количество таких перестановок, что никакой элемент не оказался на своем месте.
Приближенно равно $e^{-1}$ по формуле Тейлора, обрезанная по n-му члену(причем сходится довольно быстро), т.е. отличается не больше чем на 1/(n+1)! Т.е. точность данной формулы достаточно высокая.

Т.е. ответ "задачи о беспорядках" - $1/e \approx 0,368$

\subsection{Пример. Задача о встречах}

Продолжение предыдущей задачи.

Мы хоим узнать, сколько таких перестановок, что ровно r элементов остались на своих местах.

Задача сводится к предыдущей:

Выберем r индексов из n элементов  - $C_n^r$. они зафиксированы, а на оставшихся местах - элементы в беспорядке.

Мы знаем количество беспорядкаов на остальных местах - нужно просто оборвать формулу из предыдущей задачи в определенный момент:
$$
C_{n}^{r}(n-r)!(1-1+1/2!-1/3! + \ldots + (-1)^{n-r} \frac{1}{(n-r)!} ) = 
\frac{n!}{r!} (1-1+1/2!-1/3! + \ldots + (-1)^{n-r} \frac{1}{(n-r)!} ) \approx \frac{n!}{r!e}
$$
Из этой формулы, например, можно посчитать вероятность того, что ровно один элемент наодится на своем месте. Она практически не отличаются от вероятности, когда ни один элемент не находтся на своем месте, т.е. разница исчезающе мала для больших $n$.

\subsection{Функция Эйлера}

Функция Эйлера --- количество чисел от 1 до $n$ взаимно простых с $n$.
Формула этой функции также находится с помощью формулы включений-исключений.

Пусть $n$ раскладывается на множители следующим способом:
$$
n = p_1^{{\alpha}_1} p_2^{{\alpha}_2} \ldots p_k^{{\alpha}_k}
$$
где $p_i$ - некие простые числа

Свойство, которое мы будем считать - это делимость числа m на $p_i$

Соответственно, мы будем интересоваться чисел, которые не делятся ни на одну из $p_i$

Т.е., нужно сначала найти, сколько чисел от 1 до n делатся на $p_{i_1}, p_{i_2},  \ldots ,p_{i_j}$

Числа, которые просто кратны p и не превосходящие $n$:
$$
p, 2p, \ldots, np/p 
$$

Их $n/p$ штук.

Аналогично количество чисел, не превосходящих $n$ и кратных 
$p_{i_1}$, $p_{i_2}$, \dots, $p_{i_j}$, равно
$$
\frac{n}{p_{i_1} p_{i_2} \ldots p_{i_j}}
$$

Поэтому формула ВИ

$$
\phi(n) = n - \sum\limits_{i_1} \frac{n}{p_i} + 
\sum\limits_{i_1 < i_2} \frac{n}{p_{i_1} p_{i_2}} 
- \sum\limits_{i_1 < i_2 < i_3}^{} \frac{n}{p_{i_1} p_{i_2} p_{i_2}} 
+ \ldots 
+ (-1)^{k} \frac{n}{p_1 p_2 \ldots p_k} 
= n(1 - 1/p_1)(1 - 1/p_2) \ldots (1 - 1/p_k)
$$

Выкладки по сокращению опущены.

Вкратце, выражение разворачивается в приведенное, т.к. все одиночные произведения  - со знаком минус, 
все попарные произведения  - со знаком минус, все по три - со знаком плюс и т.д.


\section{Разбиения.}

\subsection{Разбиение множеств}

Есть множество из n элементов (не обязательно чисел)
$$
\{a_1, a_2, \ldots, a_n\}
$$

необходимо узнать количество способов разбить его на множество из 
$k_1, k_2, k_3, \ldots, k_r$ элементов, так что $k_1+k_2+k_3+ \ldots+k_r = n$

Честный случай, когда у нас разбиение на 2 подмножества:

$k_1 + k_2 = n$

Тогда это биномиальный коэффициент $C_n^{k_1}$

Теперь посчитаем в общем случае. Если мы хотим разбить на произвольное количество элементов, 
последователбно выберем $k_1, k_2, k_3, \ldots, k_r$ элементов.

Сначала выберем $k_1$ элементов, количество способов сделать это $C_n^{k_1}$

Затем выберем $k_2$ элементов, количество способов сделать это $C_{n-k_1}^{k_2}$

И т.д. до r

Общее количество способов
$$
C_{n}^{k_1} C_{n-k_1}^{k_2} C_{n-k_1-k_2}^{k_3} \ldots C_{n-k_1-k_2- \ldots - k_{r-1}}^{k_r}
$$

Распишем по факториалам
$$
= \frac{n!}{(n-k_1)!k_1!} \frac{(n-k_1)!}{(n-k_1-k_2)!k_2!} \frac{(n-k_1-k_2)!}{(n-k_1-k_2-k_3)!k_3!} 
\ldots  \frac{(n-k_1-k_2- \ldots - k_r)!}{(n-k_1-k_2-k_3- \ldots - k_r)!k_r! 0!} 
$$

Числители и знаменатели попарно сокращаются, остается

$$
\frac{n!}{k_1! k_2! \ldots k_r!}
$$

Обратите внимание, что это обобщение биномиального коэффициента при $(r=2)$ .

Поэтому этот объект обычно называется \textbf{мультиномиальным коэффициентом}. Обозначается он как
$$
\binom{n}{k_1, k_2, \ldots, k_r}
$$


\begin{description}
\item[Пример. Раскладки в преферансе]~	

В преферансе раздается 32 карты, трем игрокам по 10 карт и 2 в прикупе. Порядок получения карт не имеет значения, порядок игроков разумеется существенен. Общее количесво раскладок будет $ \frac{32!}{10!10!10!2!} \approx 2,75 \cdot 10^{15} $ \\
Т.е. шансов повторения игры - практически никаких.

\end{description}

\begin{description}
\item[Пример]~	

Возьмем следующее выражение:
$$
(x_1 + x_2 + \ldots + x_r)^n
$$

Раскрыв скобки, получим сумму некоторых мономов:
$$
\sum x_1^{k_1} x_2^{k_2} \ldots x_r^{k_r}
$$

Для того, чтобы возникло $x_1^{k_1}$ мы должны в $k_1$ скобок выбрать первый элемент (т.е. отнести $k_1$ скобок в первое множество). \\
Для того, чтобы возникло $x_2^{k_2}$ мы должны в $k_2$ скобок выбрать второй элемент. \\
И т.д. \\
$\ldots$
Т.е. это то же самое, что взять и разбить множество на ${k_1! k_2! \ldots k_r!}$ элементов.
Т.е. перед каждый элемент суммы определяется мультиномиальным коэффициентом.
$$
\sum \binom{n}{k_1, k_2, \ldots, k_r} x_1^{k_1} x_2^{k_2} \ldots x_r^{k_r}
$$

Например при $k=2, r=2$ получим обычный бином.
$$
(x_1 + x_2)^2 = \sum \binom{2}{k_1=2, k_2=0} x_1^{k_1} x_2^{k_2} 
+ \sum \binom{2}{k_1=1, k_2=1} x_1^{k_1} x_2^{k_2}
+ \sum \binom{2}{k_1=0, k_2=2} x_1^{k_1} x_2^{k_2}
$$
Т.е., хорошо известная формула квадрата суммы.
\end{description}

\subsection{Задача о разбиении чисел}

В задаче о разбиении чисел имеется 2 варианта, и первый намного проще с точки зрения выяснения его свойств.

Пусть имеется число n, и мы выясняем количество способов представить его в виде суммы k слагаемых.
$$
n = x_1 + x_2 + \ldots + x_k
$$

\subsection{Задача об упордоченных разбиениях}

В первом варианте порядок разбиения имеет значение (мы считаем упорядоченные разбиения)

Т.е.
$$
3 = 2+1 \\
3 = 1+2
$$
- разные разбиения

Если мы интересуемся упорядоченными разбиениями, их можно представить следующим образом:

Обозначим число n как n точек. В промежутки между n точками поместим k-1 перегородок.
Всего перегородок может быть от 0 до n-1.
Т.к. нулей нет, по краям перегородка ставить нельзя (несколько в ряд тоже нельзя)

TODO картинка

Каждой расстановке перегородок соответсвует разбиение.
Таким оразом мы получили биекция между количесовом разбиений n чисел на k слагаемых и количеством расстановки перегородок.

Это количество нам известно, по определению оно равно 
$$
C_{n-1}^{k-1}
$$

TODO А что если количество перегородок = 0 \\

Если же мы хотим узнать количество всех возможных разбиений от 0 до n-1, это количество тоже изветсно
$$
2^{n-1}
$$
Получить его можно, просто сложив все биномиальные коэффициенты, или вспомнив, что это количество всех подмножеств данного множества (т.е., возможностей расставить хоть сколько-нибудь перегородок)

\subsection{Задача о неупордоченных разбиениях}

Если мы представляем в виде суммы 
$$
n = x_1 + x_2 + \ldots + x_k
$$

Такое количество разбиений обозначим $p_k(n)$

В данном случае считаем, что
$$
3 = 2+1
3 = 1+2
$$
- это одно и то же разбиение

Для того чтобы зафиксировать различные $x_n$, будем считать, что они не возрастают
$$
x_1 \ge x_2 \ge \ldots \ge x_k
$$

Теперь у нас есть только одно разбиение.

Заведем рекуррентную формулу:
$$
n = x_1 + x_2 + \ldots + x_k
$$

Вычтем из каждого x по 1
$$
n-k = (x_1 - 1) + (x_2 = 1) + \ldots + (x_k - 1)
$$

И обозначим каждую скобку как $y_i$
$$
n = y_1 + y_2 + \ldots + y_k
$$

Для всех y также выполняется условние монотонности
$$
y_1 \ge y_2 \ge \ldots \ge y_k
$$

и возможна ситуация, когда для какого-либо $y_s \ge 0$, а $y_{s+1} = 0$, тогда все последующие тоже равны 0:
$$
y_{s+1} = y_{s+1} = \ldots = y_k 0
$$

Соответственно, все 
$$
x_1 \ge x_2 \ge \ldots \ge x_s \ge 1
$$

Таким образом, получаем разбиение числа n на в точности s слагаемых, 
а т.к. $0 \ge s \ge k$, получим разбиение не более чем на k слагаемых (потому что какие-то их y могут обратиться в 0).

Такое количество несложно посчитать, просто сложив количество разбиений на 1 слагаемых, на 2 и т.д.

$$
p_k(n) = p_{k}(n-k) + p_{k-1}(n-k) + \ldots + p_{1}(n-k)
$$
Здесь $p_k(n)$ - количество способов разбить число n ровно на k слагаемых

Теперь внимательно посмотрим на сумму начиная со второго слагаемого
$$
p_k(n) = p_{k}(n-k) \underbrace{+ p_{k-1}(n-k) + \ldots + p_{1}(n-k)}_{p_{k-1}(n-1)}
$$

по той же формуле.

Таким образом, получим рекуррентное соотношение:
$$
p_k(n) = p_{k}(n-k) + p_{k-1}(n-1)
$$

TODO Спросить про таблицу.

Очевидно, при $n=k$ количество разбиений равно 1 (и даже при $n=k-1$ количество разбиений равно 1)

TODO соотношения для k=2 k=3 \\

Можно показать, что $p_k(n)$ является многочленом степени $k-1$ со старшим членом равным 
$$
\frac{n^{k-1}}{(k-1)!k!}
$$
, а коэффициенты $p_2(n), p_2(n), \ldots$ зависят от остатка $n$ по $ \mod k!$ (примем без доказательства)

По этой формуле удобно считать количество разбиений только для небольших номеров и n больше k.

\subsection{Соотношения между упорядоченными и неупордоченными разбиениями}

Посчитаем примерно, во сколько раз \textit{примерно} упорядоченных разбиений больше чем неупорядоченных.

Неупорядоченные разбиения (берем только первый член):
$$
p_k(n) = \frac{n^{k-1}}{(k-1)!k!} + \ldots
$$

Упорядоченные разбиения:
$$
C_{n-1}^{k-1} = \frac{(n-1)!}{(n-k)!(k-1)!}
$$

Составим отношение:
$$
\frac{C_{n-1}^{k-1}}{p_k(n)} = \frac{n^{k-1}}{(k-1)!k!} \frac{(k-1)!(n-k)!}{(n-1)!} =
$$
сокращаем $(k-1)!$

$$
= \frac{n^{k-1}}{(n-1)(n-2) \ldots (n-k+1) k!} = 
$$

Поделим каждую скобку почленно на n. \\
Теперь заметим, если n очень большое, то числитель стремится к 1.
$$
= \underbrace{\frac{1}{(1-1/n)(1-2/n) \ldots (1-(k+1)/n)}}_{ \lim_{n \to \infty} =1 }  \frac{1}{k!} = \frac{1}{k!}
$$

Вывод отсюда следует такой - неупорядоченному разбиению соответствет примерно $1/k!$ упорядоченных, а это просто число перестановок, а это означает, что в подавляющем большинстве разбиений слагаемые \textbf{различны}. 
Справедливо, если $n \gg k$

\subsection{Разбиение на нефиксированное число слагаемых}

Обозначается $p(n)$
$$
n = x_1 + x_2 + \ldots + x_k
$$
также будем считать, что $x$ не возрастают
$$
x_1 \ge x_2 \ge \ldots \ge x_k
$$

Получаем просто сумму 
$$
p(n) = p_1(n) + p_2(n) + \ldots + p_k(n)
$$

TODO спросить таблицу

Как видно из таблицы, скорость роста довольно большая.
Приведем еще больше значений, чтобы убедиться в этом.

TODO спросить еще таблицу

При первичной оценке $p(n)$ приближенно ведет себя как $e^{ \sqrt{n} }$

Для поиска p(n) сложно использовать биекцию (скорее сюрьекцию или иньекцию), т.е. получается не точное равенство, а некоторая оценка.
Также для их поиска используются производящие функции.

\section{Диаграммы Юнга}

TODO Нарисовать диаграммы (заметить, что пропущено разбиение 5=5) \\

$$
p(1) = 1
$$

$$
p(2) = 2
$$

$$
p(3) = 3
$$

$$
p(4) = 5
$$

$$
p(5) = 7
$$

$$
p(6) = 11
$$

Многие доказательства, связанные с разбиениями, основаны на подобных графических представлениях.

Также для представления разбиений используются \textbf{графы (диаграммы) Ферре} (просто используются точки вместо квадратов)

TODO рисунок

\subsection{Теоремы, доказываемые с помощью диаграмм}

\begin{description}
\item[Теорема]~	

Число разбиений $ n $ на $ k $ частей равно числу разбиений на части, наибольшая из которых равна $ k $.

Доказательство проводится чисто из визуальных соображений.

TODO рисунок

Рисуем диаграмму Юнга.
$$
k = 5, n = 10
$$

Из количества разбиений на $ k $ частей получить кол-во разбиений, наибольшая из которых равна $ k $, диаграмму можно взять и симметрично отразить относительно диагонали (такая диаграмма называется \textit{сопряженной}).

Доказательство очевидно из рисунка

TODO Рисунок
TODO

\end{description}


\begin{description}
\item[Теорема]~	

Число разбиений $ a-c $ в точности на $ b-1 $  слагаемое (такое что $ b \le c $), равно количеству 
разбиений числа $a-b$ в точности на $c-1$ слагаемое, каждое из которых не превосходит $b$.

TODO диаграммы Юнга (или Ферре)

$$
b = 5, c = 7, a - c = 13
$$

Пририсуем сверху (снизу) строчку из $c$ элементов.

Получим разбиение $c$ на $b$ элементов (т.к. перед добавлением там был $b-1$ элемент).

Вычеркнем этот столбец, и возьмем сопряженную диаграмму.

Т.е. мы получили разбиение на $c-1$ слагаемое $a_i$, каждое из которых $ a_i \le b$

\end{description}


\begin{description}
\item[Теорема Эйлера]~	

Рассматриваем разбиения только на слагаемые, принадлежащие какому-либо классу, например, только на нечетные, или только на различные.

TODO таблица

Интересно, что по крайней мере для 7 (а на самом деле и дальше) количество разбиений только на нечетные или только на различные, равно.

Между ними можно построить биекцию.
Возьмем разбиения на \textbf{различные} слагаемые. Выбыраем из них все четные, и разложим на одинаковые нечетные слагаемые, являющиеся максимальной степень двойки.
Т.е.
$$
2^a b = \underbrace{b + b + \ldots +b}_{2^a}
$$

Например, $12 = 3+3+3+3$

Таким образом, мы получили биекцию между разбиениями.

Но оказывается, существует и обратное преобразование (нечетные в различные).
Если есть нечетное число $ b $, встречающееся $ k $ раз, то из него можно восстановить различные
Запишем k в двоичной системе счисления.
$$
k = 2^0 + 2^{k_1} + 2^{k_2} + \ldots + 2^{k_n}
b + 2^{k_1} b + 2^{k_2} b +\ldots + 2^{k_n} b
$$
Например:
$$
k = 2^0 + 2^3 + 2^{10} + 2^{11}
b + 8 b + 1024 b + 2024 b
$$
Очевидно, все слагаемые будут различны, т.к. никакая степень двойки в двоичной записи 2 раза не встречается (не может дважды встретится $ 2^3 $, вместо этого мы записали бы  $ 2^6 $).


\end{description}





