\section{Числовые последовательности}


\subsection{Числа Каталана}

Пусть имеется правильная скобочная последовательность: \\
\textbf{(()())()(())}

Причем так, что в каждый момент открывающихся скобок не менее чем закрывающихся, 
т.е. можно разбить на пары \textbf{()}, которые не пересекаются.

Найдем рекуррентное соотношение и производящую формулу для подобных скобочных структур.

пусть $ C_{n} $ - количество правильных скобочных структур для $ 2n $ скобок 
($ n $ открывающихся, $ n $ закрывающихся)

Считаем, что $ C_{0} = 1 $

Выразим $ C_{n+1} $

Для каждой открывающей скобки есть соответствующая ей открывающая. 
Тогда если мы воззьмем по отдельности этот набор(0), оставшийся после него(1), и оставшийся после удаления из него двух крайних скобок(2).

$ \overbrace{( \underbrace{(()()())}_{(1)} )}^{(0)} \overbrace{(()())}^{(2)} $

Все они будут правильными скобочными последовательностями.

Если в (1) $ k $ открывающихся скобок, то в (2) $ n-k $ открывающихся скобок.

Т.е. в (1) имеется $ C_{k} $ способов расставить скобки, а в (2) - $ C_{n-k} $

Во всей конструкции получается $ C_{k} C_{n-k} $ способов.
А т.к. разбить конструкцию на (1) и (2) мы можем в любом месте,
для получения рекуррентного соотношения надо просто сложить 
все возможные случаи разбиений

$$
C_{n+1} = C_0 C_n + C_1 C_{n-1} + C_2 C_{n-2} + \ldots + C_{n-1}C_1 + C_n C_0
$$

$$
C_{n+1} =  \sum\limits_{i = 0}^{n} C_i C_{n-i}
$$

А начальное число $ C_{0} = 1 $

Последовательность, удовлетворяющая приведенному рекуррентному соотношению,
называется \textbf{числами Каталана} 

%TODO таблица

Как видно, числа растут очень быстро.

\subsection{Комбинаторные представление чисел Каталана.}

\subsubsection{Триангуляции}

Рассмотрим некий $n+2$-угольник, и рассмотим все его триангуляции.

%TODO картинка

Оказывается, количество триангуляций $n+2$-угольника является в точности $n$-м числом Каталана.

Это верно, т.е. между триангуляциями и скобочными последовательностями существует взаимно однозначное соответсвтие (при разбиениеии $ n+2$-угольника получаем 
с одной стороны $ C_{k} $ триангуляций, с другой - $ C_{n-k} $ триангуляций).

\subsubsection{Непересекающиеся хорды на окружности}

%TODO картинка

Для 6 точек - 5 способов провести непересекающиеся хорды.

Соответственно, для $ 2n $ точек - $ n $ способов провести непересекающиеся хорды.

При разбиениеии $2n+2$-точечной окружности получаем с одной стороны $ C_{k} $ 
способов провести хорды, с другой - $ C_{n-k} $ способов).

\subsubsection{Графы без циклов}

%TODO картинка

Количество графов без циклов (неизоморфных корневых деревьев) с $ n+1 $ узлами
также является $n$-м числом Каталана.

Пусть имеется одна вершина (корень).
Всего $ n+2 $ вершины.
Произведем разбиение, что удаляется одно из ребер 
таким образом, что в оставшемся дереве остается $ k+1 $ вершина с одной стороны, 
а с другой - $ n - k + 1 $.
Тогда с одной стороны имеется $ C_{k} $ вариантов, с другой - $ C_{n-k} $ вариантов

\subsection{Производящая функция чисел Каталана}

Производящая функция выглядит так:

$$
Cat(t) = c_{0} + c_{1} t^{} + c_{2} t^{2} + \ldots 
$$

Возведем в квадрат:

$Cat(t)^{2} = c_0^2 + (c_1 c_0 + c_0 c_1)t +(c_2 c_0 + c_1 c_1 + c_0 c_2)t^2
+ (c_3 c_0 + c_2 c_1 + c_1 c_2 + c_0 c_3)  t^3 + \ldots $

Теперь воспользуемся рекуррентной формулой:

$ c_{1}= c_{2} t^{} + c_{3} t^{2} + \ldots $

Составим следующее соотношение.
$$
t Cat(t)^2 = Cat(t)-1
$$

Очевино, это квадратное уравнение относительно функции Каталана

$$
Cat(t) = \dfrac{1 \pm \sqrt{1 - 4 t}}{2 t}
$$

Правильный знак '-', иначе в 0 объект не определен. 

$$
\sum_{n=0}^{\infty} C_n t^n = \frac{1-\sqrt{1-4 t}}{2 t}
$$

Напишем формулу Тейлора для этого объекта:

$(1+x)^p = 1 + px + \dfrac{p(p-1)}{2!} x^{2} + \dfrac{p(p-1)(p-2)}{3!} x^{3} + \ldots$

$\dfrac{1-\sqrt{1-4 t}}{2 t} = \frac{1}{2t} (1 - (1-4 t)^{1/2} )$

Распишем $ (1-4 t)^{1/2} $ по формуле Тейлора:

$(1-4 t)^{1/2} = $ \\
$\dfrac{1}{2t}$
$ ( 1 -  $
% разложение в ряд Тейлора начинается здесь
$(1 $
$ + \dfrac{1}{2}(-4t) + $ \\ \\
$ + \dfrac{1/2 (1/2 - 1)}{2!} (-4t)^{2} + $ \\ \\
$ + \dfrac{1/2 (1/2 - 1)(1/2 - 2)}{3!} (-4t)^{3} + $ \\ \\
$ + \dfrac{1/2 (1/2 - 1)(1/2 - 2)(1/2 - 3)}{3!} (-4t)^{4} $ 
$ + \ldots ) = $ \\

Знак при каждом слагаемом всегда будет '+'

$ = \dfrac{1}{2t}(2t + $
$ + \dfrac{2^{2}}{2!}t^{2}  $
$ + \dfrac{2^{3}}{3!} 1\cdot3 t^{3}  $
$ + \dfrac{2^{4}}{4!} 1 \cdot 3 \cdot 5 t^{4} ) + \ldots 
+ \dfrac{2^{k}}{k!} 1 \cdot 3 \cdot 5 \cdot \ldots (2k-3) t^{k} ) + \ldots = $

Сократим на $ 2t $. 
Кроме того, произведение всех нечетных чисел называется "двойной факториал",
и обозначается $ n!! $

$ = (1 + \dfrac{2 t}{2!} t  $
$ + \dfrac{2^{2}}{3!} 3!! t^{2}  $
$ + \dfrac{2^{3}}{4!} 5!! t^{3} ) + \ldots $
$+ \dfrac{2^{k-1}}{k!} (2k-3)!! $

Таким образом, формула для $ n-$го числа Каталана

$$
c_n = \frac{2^n}{(n+1)!} (2n-1)!! = \frac{C_{2n}^n}{n+1} 
$$

Докажем последнее равенство:

$\dfrac{C_{2n}^n}{n+1}  = $
$\dfrac{(2n)!}{(n+1)n!n!} = $
$\dfrac{(2n)!}{(n+1)!n!} = $ \\

Вынесем из числителя $ (2n)! $ все нечетные множители - 
получим произведение всех четных на все нечетные, т.е. на $ (2n-1)!! $

$= \dfrac{2 \cdot 4 \cdot 6 \ldots 2n}{n!} (2n-1)!!$ \\

В числителе все четные от 2 до $ 2n $

В знаменателе произведение всех чисел от 1 до $ n $

При взаимном сокращении получается как раз $ 2^{n} $

Т.е. имеет место формула
$$\dfrac{2 \cdot 4 \cdot 6 \ldots 2n}{n!} = 2^{n}
$$ 

Итоговую формулу можно записать и так:
$$
c_n = \dfrac{C_{2n}^n}{n+1} = C_{2n}^n - C_{2n}^{n-1}
$$

Последнее равенство показывает, что формула всегда дает целое число.

Докажем, почему это верно :
$$
\dfrac{C_{2n}^n}{n+1} = C_{2n}^n - C_{2n}^{n+1}
$$
$$
\dfrac{(2n)!}{(n+1)n!} = \dfrac{(2n)!}{n!n!} - \dfrac{(2n)!}{(n+1)(n-1)!}
$$

Делим на $ (2n)! $, домножаем на $ (n+1)n! $

$$
1 = n + 1 - n
$$

Доказано

\subsection{Асимптотика для чисел Каталана}


Скорость, с которой растут числа Каталана:

$$
\dfrac{C_{2n}^n}{n+1} 
\sim \dfrac{1}{n + 1} \dfrac{4^{n}}{\sqrt{\pi n}} 
\sim \dfrac{4^{n}}{\sqrt{\pi} n^{3/2}}
$$

\section{Числа Стирлинга}

\subsection{Числа Стирлинга второго рода. Комбинаторный смысл}

числом Стирлинга второго рода из n по k, обозначаемым S(n,k), или
$\begin{Bmatrix}
n \\
k
\end{Bmatrix}$
, называется количество неупорядоченных разбиений n-элементного множества на k непустых подмножеств.


Берем множество из $ n $ элементов 
$$
1,2,3,\ldots,n
$$
и считаем количество способов
разибть его на $ k $ непустых подмножеств.

%TODO спросить про пример

Пример:

$\begin{Bmatrix}
4 \\
2
\end{Bmatrix} = 7$

%TODO спросить про таблицу

Очевидно, количество способов разбить $ n $ на 1 равно 1 (первый столбец)

Также, количество способов разбить $ n $ на $ n $ равно 1 (диагональ)

На второй диагонали - количество способов разбить $ n $ на $ n-1 $ непустое подмножество. 
Т.е., все подмножества, ктоме одного - одноэлементны, значит, это количество способов выбрать 2-элементное подмножество, а именно $ C_{n}^{2} $

Довольно сложно устроено общее количество способов разбить $ n$-элементные множества на непустые подмножества (сумма по строкам).

Носит название \textbf{чисел Белла}, и растет очень быстро.

\subsubsection{Рекуррентная формула}

Возьмем число 
$\begin{Bmatrix}
n \\
k
\end{Bmatrix}$

Возьмем элемент с номером $ n $.

1 вариант. Будем считать его отдельным подмножеством, тогда останется
$\begin{Bmatrix}
n-1 \\
k-1
\end{Bmatrix}$ способов.

2 вариант. Не будем считать его отдельным подмножеством 
(он содержится в подмножестве), тогда останется
$k \begin{Bmatrix}
n-1 \\
k
\end{Bmatrix}$, где множитель $ k $ - количество способов включить 
этот элемент в любое из подмножеств.

Сведем в рекуррентную формулу:

$$
\begin{Bmatrix}
n \\
k
\end{Bmatrix} = 
\begin{Bmatrix}
n-1 \\
k-1
\end{Bmatrix} +
k \begin{Bmatrix}
n-1 \\
k
\end{Bmatrix}
$$

\subsection{Числа Стирлинга первого рода}

Числа Стирлинга первого рода --- количество перестановок порядка n с k циклами.

Обозначается
$\begin{bmatrix}
n \\
k
\end{bmatrix}$

Берем множество из $ n $ элементов 
$$
1,2,3,\ldots,n
$$
и разбиваем его не на подмножества, а на непересекающиеся циклы.

Например,
$(a,b,c),(b,c,a),(c,a,b)$ - циклы.

%TODO спросить про пример
\textbf{TODO спросить про пример}

Пример:

Т.е. $ \{1,2,3\} $ и $ \{1,3,2\} $ - разные циклы, но одинаковые множества.

А $ \{1,2\} $ и $ \{2,1\} $ - одинаковые множества и циклы.

Понятно, что чисел 1-го рода больше, чем 2-го, т.к. одному множеству 
может соответствовать много циклов.

А именно, множеству из $ k $ элементов, бeдет соответсвовать $ (k-1)! $ циклов 
(как соотношение упорядоченных и неупорядоченных множеств)

$\begin{Bmatrix}
4 \\
2
\end{Bmatrix} = 11$


%TODO спросить про таблицу
\textbf{TODO спросить про таблицу}
Ясно, что количество способов разбить $ n $ на $ n $ циклов всего один (первая диагональ).

Количество способов разбить на 2 цикла всего равно количеству 
разбить на 2 множества (вторая диагональ).

Количество способов сделать из числа цикл - $ (n-1)! $ (первый столбец)

Количество способов сделать из числа любое число циклов - $ n!$ (сумма по строкам)

\subsubsection{Рекуррентная формула}

Возьмем элемент с номером $ n $.

1 вариант. Будем считать его образующим собственный цикл, 
тогда останется $ n-1 $ элемент, из них нужно сделать $ k-1 $ циклов
$\begin{bmatrix}
n-1 \\
k-1
\end{bmatrix}$ способов.

2 вариант. Не будем считать его отдельным циклом
(он содержится в цикле), тогда останется
$ (n-1) \begin{bmatrix}
n-1 \\
k
\end{bmatrix}$, где множитель $ (n-1) $ - количество способов включить 
этот элемент в любqой из циклов (т.е. в любое место между $ n $ элементами).

Сведем в рекуррентную формулу:

$$
\begin{bmatrix}
n \\
k
\end{bmatrix} = 
\begin{bmatrix}
n-1 \\
k-1
\end{bmatrix} +
(n-1) \begin{bmatrix}
n-1 \\
k
\end{bmatrix}
$$

Совпадает во всем с числами Стирлинга 2-го рода, кроме множителя при $\begin{bmatrix}
n-1 \\
k
\end{bmatrix}$.

\subsection{Применение}

\subsubsection{Пример 1}

Возьмем некий многочлен n-й степени, который можно записать как сумму коэффициентов (минимальный коэффициент - 1).

Сумма коэффициентов при $ x^{k} $ - числа Стирлинга 1-го рода.

$x(x+1)(x+2)(x+3)\ldots(x+n-1) = \sum\limits_{k=1}^{n} \begin{bmatrix}
n-1 \\
k
\end{bmatrix} x^{k}$

Доказательство по инфукции:

Домножим выражение на $ (x+n) $:

$x(x+1)(x+2)(x+3)\ldots(x+n-1)(x+n) = $ \\
$= \sum\limits_{k=1}^{n} \begin{bmatrix} n-1 \\ k \end{bmatrix} x^{k} (x+n) = $\\

%TODO непонятно, откуда следует
\textbf{TODO непонятно, откуда следует}

$= \sum\limits_{k=1}^{n} 
\left(
\begin{bmatrix} n-1 \\ k \end{bmatrix} x^{k+1} + n \begin{bmatrix} n \\ k \end{bmatrix} x^{k} = 
\right)
$

$= \sum\limits_{k=1}^{n+1} 
\left(
\begin{bmatrix} n \\ k-1 \end{bmatrix} x^{k} 
+ n \begin{bmatrix} n \\ k \end{bmatrix} x^{k}
\right) = 
$

$= \sum\limits_{k=1}^{n+1} 
\begin{bmatrix} n+1 \\ k \end{bmatrix} x^{k} 
$

\subsubsection{Пример 2}

Поставим обратную задачу - пусть у нас есть $ x^{n} $, как его выразить через последовательные произведения.

\textbf{Примечание}

Краткое обозаняение для последовательного произведения вида
$$
x(x-1)(x-2)(x-3)\ldots(x-n+1) = x^{\underline{n}}
$$

, а для произведения вида
$$
x(x+1)(x+2)(x+3)\ldots(x+n-1) = x^{\overline{n}}
$$

Таким образом, 

$$
x^{\overline{n}} = \begin{bmatrix} n \\ k \end{bmatrix} x^{k} 
$$

А для обратной задачи верно следующее:
$$
x^{n} = \begin{Bmatrix} n \\ k \end{Bmatrix} x^{\underline{k}} 
= \begin{Bmatrix} n \\ k \end{Bmatrix} (-1)^{n+k} x^{\overline{k}} 
$$

Числа Стирлинга 1 и 2 рода обычно и применяются для связи 
подобных последовательных произведений со степенными выражениями.

\section{Числа Белла}

Числа Белла --- количество спобобов разбить на любые подмножества n-элементное множество.

$$
b_{n} = \sum\limits_{k=1}^{n} \begin{Bmatrix} n \\ k \end{Bmatrix}
$$

\subsection{Рекуррентная формула}

Выделим последний элемент $ n+1 $
Он входит в какое-либо подмножество.
Количество способов добавить к ним этот элемент и еще $ k $ элементов - $ C_{n}^{k} $
Т.е., создано подмножество, в которое заведомо входит элемент $ n+1 $

Количество спобобов разбить оставшиеся элементы на любые непустые подмножества.
Это $ n-k$-е число Белла.
Осталось просуммировать по $ k $ и свести в рекуррентную формулу.

\begin{equation}
\label{bell_recur}
b_{n+1} = \sum\limits_{k = 0}^{n} C_{n}^{k} b_{n-k} = \sum\limits_{k = 0}^{n} C_{n}^{k} b_{k} 
\end{equation}

\subsection{Экспоненциальная производящая функция}

Есть последовательность чисел
$$
a_0, a_1, a_2, \ldots 
$$

и формальный степенной ряд
$$
A(t) = a_0 + \dfrac{a_1}{1!}t + \dfrac{a_2}{2!}t^{2} + \dfrac{a_3}{3!}t^{3} +\ldots 
$$

Если все $ a_i $ равны 1, то ряд сходится к $ e $

Если перемножить два ряда:
$A(t) = a_0 + \dfrac{a_1}{1!}t + \dfrac{a_2}{2!}t^{2} + \dfrac{a_3}{3!}t^{3} +\ldots $
$B(t) = b_0 + \dfrac{b_1}{1!}t + \dfrac{b_2}{2!}t^{2} + \dfrac{b_3}{3!}t^{3} +\ldots $

То коэффициенты произведения
$A(t) B(t) = c_0 + \dfrac{c_1}{1!}t + \dfrac{c_2}{2!}t^{2} + \dfrac{c_3}{3!}t^{3} +\ldots $

устроены следующим образом:

$\dfrac{c_n}{n!} = \dfrac{a_{0}}{0!}\dfrac{b_{n}}{n!}
+ \dfrac{a_{1}}{1!} \dfrac{b_{n-1}}{(n-1)!} 
+ \dfrac{a_{2}}{2!} \dfrac{b_{n-2}}{(n-2)!}
+ \ldots 
+ \dfrac{a_{n}}{n!} \dfrac{b_{0}}{0!} $

Разделив обе части выражения на $ n! $, получим коэффициенты - сочетания $C_{n}^{k}$
\begin{equation}
\label{exponent_gen_func}
{c_n} = C_{n}^{0} a_{0} b^{n} 
+ C_{n}^{1} a_{1} b^{n-1} 
+ C_{n}^{2} a_{2} b^{n-2} 
+ \ldots
+ C_{n}^{n} a_{n} b^{0} 
\end{equation}


\subsection{Производящая функция чисел Белла}

Внесем полученную производящую функцию в формулу (\ref{exponent_gen_func}):
$$
b_{n+1} =  \sum\limits_{k = 0}^{n} C_{n}^{k} b_{k} 
$$

Отличается она от (\ref{exponent_gen_func}) 
отсутствием коэффициента $ a $, т.е. от равен $ 1 $,
функция $ A(t) = e^{t} $, а произведение $ A(t)B(t) $ равно:
$$
A(t)B(t) = e^{t} B(t) = e^{t} \sum\limits_{n=0}^{\infty} \dfrac{b_{n+1}}{n!} t^{n}
$$

Т.е. в полученной формуле показатели степени и коэффициенты при $ b $ "опрережают" 
на 1.
Чтобы, понизить степень, возьмем производную ряда $ B(t) $

$e^{t} B(t) = B'(t)$

Получим дифференциальное уравнение 1 степени.

$e^{t} = \dfrac{B'(t)}{B(t)} = (\ln{ B(t) })' $

$\ln B(t) = e^{t} + C $

$ B(t) = e^{e^{t} + C} $

Причем константа $ C = 0 $, т.к. при t = 0 $e^{t} + C $ должен быть равнен $ 0 $.

$$
B(t) = e^{e^{t} - 1}
$$


\subsection{Формула Добинского}

$$
b_n = \dfrac{1}{e}\sum_{k=0}^\infty \dfrac{k^n}{k!}
$$

Без доказательства.