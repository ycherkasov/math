\section{Асимптотики биномиальных коэффициентов}

\subsection{Задача оценки асимптотики. О-нотация}

Хотя задача о росте биномиального коэффициента не является 
принципиально сложной, иногда возникает необходимость эту скорость
с чем-либо сравнить.

Например, что растет быстрее:

$C_n^k$ или  $4^n $ \\

$ C_n^k $ или $ \frac{4^n}{n} $

и т.п.

На практике весто точных формул обычно используются асимптотические (приближенные).

\begin{description}
\item[Напомним некоторые определения из анализа:] ~

Пусть на множстве натуральных чисел заданы отображения $ f, g $ на множество действительных чисел.

$ \exists f, g \colon \mathbb{N} \to \mathbb{R} $ 

Фуекции называются эквивалентными, если удовлетворяется следующее условие:
$$
f ~ g \Leftrightarrow \lim_{n to 0} \frac{f(x)}{g(x)} = 1 
$$

Функция $ f $ называется "о-малым" от $ g $ если выполняется следующее условие:
$$
f = o(g) \Leftrightarrow \lim_{n to 0} \frac{f(x)}{g(x)} = 0
$$
Т.е. $ f $ доминирует над $ g $ асимптотически

Функция $ f $ называется "O-большим" от $ g $ если выполняется следующее условие:
$$
f = O(g) \Leftrightarrow |f(n)| \le C|g(n)|q 
$$
Т.е. $ f $ ограничена сверху функцией $ g $ (с точностью до постоянного множителя) асимптотически.

Вышесказанное может быть справедливо не для $ n_0 $, а для некоторого $ n_j (j > 0) $.

\end{description}

\begin{description}
\item[Основные формулы О-нотации:] ~

$o(f)+o(g) = o(f)$

$o(f) \cdot o(g) = o(f \cdot g)$

$o(o(f)) = o(f)$

$o(f)+O(f) = O(f)$

$o(f) \cdot O(f) = O(f \cdot g)$

$O(f) \cdot O(g) = O(f \cdot g)$

$O(o(f)) = o(f)$
\end{description}

\begin{description}
\item[Примеры:] ~
$$
n^2 + n + 1 \sim n^2 - 1
$$
$$
n^2 + n + 1 = o(n^3)
$$
$$
n^2 + n + 1 = O(n^2)
$$
$$
e^x = 1 + x + x^2/2 + O(x^3),  x \to 0
$$

\end{description}


\subsection{Простейшие оценки}

Начнем с простейших оценок - как растет биномиальный коэффициент $ C_{n}^{k} $ при росте $ k $?
Для четных коэффициентов:

$$
C_{2n}^{0} < C_{2n}^{1} < C_{2n}^{2} < \cdots < \underbrace{C_{2n}^{n+1}}_{\max}  > C_{2n}^{n+2} > \cdots C_{2n}^{2n}
$$
Очевидно, биномиальные коэффициенты растут от 0-го до среднего коэффициента, а дальше они убывают.
Это можно понять как из комбинаторных соображений, так и используя свойство $ C_{n}^{k} = C_{n}^{n-k} $

Для нечетных коэффициентов:
$$
C_{2+1}^{0} < C_{2n+1}^{1} < C_{2n}^{2} < \cdots < \underbrace{C_{2n}^{n} = C_{2n}^{n+1}}_{max}  > C_{2n}^{n+2} > \cdots C_{2n}^{2n}
$$
В данном случае 2 самых больших биномиальных коэффициента

\subsection{Теорема}
\begin{equation}
\label{binomial1}
\frac{4^n}{2n+1} < C_{2n}^{n} < 4^n = 2^n
\end{equation}

Докажем правую часть равенства:
$4^n = 2^n$ - сумма всех коэффициентов, а $ _{2n}^{n} $ - всего лишь наибольший из них


Докажем левую часть равенства:
$4^n = 2^n$ - сумма всех коэффициентов, а $ 2n + 1 $ - их количество, т.е. получившееся выражение меньше наибольшего коэффициента.


\begin{description}
\item[Замечание 1] ~
$$
\frac{4^n}{2\sqrt{n}} < \frac{4^n}{2\sqrt{3n+1}} < \frac{4^n}{\sqrt{3n}}
$$
Левая часть доказывается по индукции, правая приводится к индукции следующим образом:
$$ 
\frac{4^n}{\sqrt{3n+1}} < \frac{4^n}{\sqrt{3n}}
$$
\end{description}



\begin{description}
\item[Замечание 2] ~
На самом дделе, более точная оценка выглядит так:
$$
\frac{4^n}{\pi \sqrt{n+1/2}} < C_{2n}^{n} < \frac{4^n}{\pi \sqrt{n+1/4}}
$$

Без доказательства (доказывается из оценки следующего интеграла, см. Фихтенгольц):
$$
\int_0^{\pi} \sin^n{x} dx
$$

\end{description}


\section{Асимптотики факториала}

\subsection{Теорема}

Также полезно оценивать $ n! $

Докажем следующее отношение:

\begin{equation}
\label{factor1}
\left( \frac{n}{e} \right)^n < n! < e (n/2)^n 
\end{equation}



Главная часть оценки в выражении уже присутствует: $ \left( \frac{n}{e} \right)^n $

Доказательство левой части: по индукции

Доказательство правой части:

Заметим, что $ n! = 1 \cdot 2 \cdot 3 \cdot \ldots \cdot n $

Перемножим их попарно, и заметим что каждое произведение меньше полусуммы в квадрате:

$ 1 \cdot n  \le \left(  \frac{n+1}{2} \right)^2 $	\\
$ 2 \cdot (n-1) \le \left(  \frac{n+1}{2} \right)^2 $ 	\\
$ 3 \cdot (n-2) \le\left(  \frac{n+1}{2} \right)^2 $	\\
$ \ldots $\\
$ n \cdot 1 \le \left(  \frac{n+1}{2} \right)^2$	\\

Перемножив почленно, получим:
$$
n!^{2}  \le \left(  \frac{n+1}{2} \right)^{2n} 
$$

$$
n!  \le \left(  \frac{n+1}{2} \right)^{n} 
$$

Осталось заметить, что $ \left(  \frac{n+1}{2} \right)^{n} <  e (n/2)^n $

Это следует из следующих соображений:
$$
\left( \frac{\frac{n+1}{2}} {e (n/2)} \right)^n = \frac{(1 + 1/n)^n}{e^n}
$$

$$ \left( \frac{n+1}{n} \right)^n = (1 + 1/n)^n < e$$

%TODO прояснить момент,это из-за того, как растет n?
\textbf{TODO прояснить момент,это из-за того, как растет n? Связано ли с 1 зам. пределом?}

\subsection{Формула Стирлинга}

Такие оценки можно достаточно качественно уточнять, но
обычно при асимптотических оценках используются не равенства, а эквивалетности, а именно используется
\textbf{формула Стирлинга}

\begin{equation}
\label{stirling}
n! \sim \sqrt{2 \pi n} \left(\frac{n}{e}\right)^n
\end{equation}

Примем без доказательства.
Факториал - это произведение, т.е. логарифм этого произведения - некая сумма. Надо сумму заменить на интеграл и оценить.

Исходя из формулы Стирлинга найдем основные эквивалетности.

$$
C_{n}^{2n} = \frac{(2n)!}{n!n!} \sim \frac{(2^n)^{2n}c^{-2n} \sqrt{2 \pi 2 n}}{ (n^n e^{-n} \sqrt{2 \pi n})^2 } =
$$
сокращаем
$$
 = \frac{ 2^{2n} \sqrt{2 \pi 2 n}}{2 \pi n} =  \frac{4^n}{\sqrt{\pi n}} 
$$

Т.е., с такой скоростью растет самый большой биномиальный коэффициент. 
Это примерно посередине между элементарными неравенствами, которые мы рассмотрели выше ($ \ref{binomial1} $).
Меньше $ 4^n $ и больше $ 4^n $ делить на какое-то выражение кратное $ n $.

Оказалось, это выражение - $ \sqrt{n} $

\section{Асимптотики факториала и биномиальных коэффициентов}

\subsection{Теорема}

\begin{equation}
\label{factor2}
C_{n}^{k} \le \frac{n^k}{k!} < \left( \frac{ne}{k} \right)^k
\end{equation}

Попробуем сначала применить оценку, только что полученную с помощью формулы Стирлинга (\ref{stirling}):
$$
C_{2n}^{n} < \left( \frac{2ne}{c} \right)^n = (2e)^{n}
$$
Плохая оценка, учитывая что $ (2e)^{n} > 4^{n} $

Докажем левую часть:
$$
C_{n}^{k} = \frac{n!}{k!(n-k)!} = \frac{n(n-1)(n-2) \ldots (n-k+1)}{k!}
$$
Т.е. каждый из сомножителей в числителе меньше n

Поэтому:
$$
C_{n}^{k} \le \frac{n^k}{k!}
$$

Докажем правую часть:

Из (\ref{factor1}) нам известно, что $k! > (k/e)^k$

Поэтому:
$$
C_{n}^{k} \le \frac{n^k}{k!}
$$

Левая часть доказана.

Докажем правую часть:

Из (\ref{factor1} ) Нам известно, что $k! > \left( \frac{k}{e} \right)^k $

Поэтому:
$$
\frac{n^k}{k!} <  \frac{n^k}{k^k} e^k = \left( \frac{ne}{k} \right)^k
$$

Правая часть доказана.

Данная оценка работает хорошо только в том случае, когда $ n $ растет намного быстее, чем $ k $,
в идеале $ k $ фиксировано.

\subsection{Теорема. Уточнение оценки}

$$
\frac{n^k}{k!}(1 - \left (\frac{k(k+1)}{2n} \right)
\le C_n^k
\le \frac{n^{k}}{k!}   e^{-\frac{k(k-1)}{2n}}
$$


Земечание: \\
Пусть $k = o(\sqrt{n}) $, тогда $ \lim \frac{k}{\sqrt{n}} = 0 $

%TODO То есть у нас n - функция и k - функция?
\textbf{TODO То есть у нас n - функция и k - функция?}

Тогда 
$$
(1 - \left (\frac{k(k+1)}{2n} \right) = (1 - \frac{o(n)}{2n}) = 1
$$

И в показателе экспоненты та же картина
$$
-\frac{k(k-1)}{2n} = 0 \Rightarrow e^{-\frac{k(k-1)}{2n}} = 1
$$

Т.е. в левой и в правой части оставется $ \frac{n^k}{k!} $

Отсюда следует:

$$
C_n^k \sim \frac{n^k}{k!} \Leftrightarrow k = o(\sqrt{n}) 
$$

Следующее наблюдение, которое можно сделать - множители при $ \frac{n^k}{k!} $
с достаточно большой точностью равны:

$$
(1 - \left (\frac{k(k+1)}{2n} \right)
\approx
e^{-\frac{k(k-1)}{2n}}
$$

что неудивительно, т.к. это первые 2 слагаемых в формуле Тейлора для экспоненты.
%TODO ORLY?
\textbf{TODO Поясните плз (желательно выкладку)}

Докажем это.
Начало формулы такое же:
$$
C_{n}^{k} = \frac{n!}{k!(n-k)!} = \frac{n(n-1)(n-2) \ldots (n-k+1)}{k!} = 
$$
поделим теперь числитель и знаменатель на $ n $
$$
 = \frac{n^k}{k!}(1 - \frac{1}{n})(1 - \frac{2}{n}) \ldots (1 - \frac{k-1}{n})
$$

Для оценки сверху вспомним соображение, что $ e^x > 1 + x, x != 0 $

Выпишем все множители:

$ (1 - \frac{1}{n}) \le e^{-1/n}  $ \\
$ (1 - \frac{2}{n}) \le e^{-2/n} $ \\
$ \ldots $ \\
$ (1 - \frac{k-1}{n}) \le e^{-(k-1)/n}  $\\

Значит, все множители вместе:
$$
(1 - \frac{1}{n})(1 - \frac{2}{n}) \ldots (1 - \frac{k-1}{n}) \le 
e^{-\frac{1}{n} - \frac{2}{n} - \ldots - \frac{k-1}{n}} = 
e^{ \frac{k(k-1)}{2n} }
$$

Таким образом получли верхнюю оценку (правая часть неравенства).

Попробуем получить нижнюю:


Воспользуемся следующим соображением:

$ (1-x)(1-y) = 1-x-y+xy > 1-x-y $ \\
$ (1-x)(1-y)(1-z) > 1-x-y-z $ \\
$ (1-1/n)(1-2/n) > 1 - \frac{1}{n} - \frac{2}{n} \ldots - \frac{k-1}{n} 
= 1 - \frac{k(k-1)}{2n}$

Таким образом получли нижнюю оценку (левая часть неравенства).

\subsection{Теорема. Асимпототики при быстром росте k}

Все предыдущие оценки сделаны из соображения, что $ k $ 
растет намного медленнее $ n $

Рассмотрим случай, когда это не так:
%TODO [n \alpha] ближе к 1?
%\textbf{TODO [n \alpha] все же ближе к 1?}

Составим следующий биномиальный коэффициент:
$$
C_n^{[n \alpha]}
$$
Где $ \alpha \in (0, 1) $ и функция  $ [n \alpha] $ - целая часть аргумента $ n \alpha $. Например $([3.4] = 3) $

Или подробнее:
$$
C_{n}^{[n \alpha]} = \frac{n!}{(n - [n \alpha])! [n \alpha]!}
$$
В этом случае $ n $ и $ n \alpha $ растут примерно одинаково.

Для этой формулы хороший результат должна дать формула Стирлинта (\ref{stirling}), которая вообще хороша для случаек, когда $ n $ и $ k $ растут примерно одинаково.

Распишем биномиальный коэфиициент по формуле Стирлинга:
$$
% огромная формула!
\frac{n!}{(n - [n \alpha])! [n \alpha]!} \sim 
% числитель
\frac
{
\overbrace{ n^{n}e^{-n}\sqrt{2 \pi n} }^{n!} 
}
% знаменатель
{
\underbrace{ [n \alpha]^{[n \alpha]} e^{-[n \alpha]} \sqrt{2 \pi [n \alpha]} }_{[n \alpha]!} 
\underbrace{ (n - [n \alpha])^{n - [n \alpha]} e^{n + [n \alpha]} \sqrt{2 \pi (n - [n \alpha])} }_{ (n - [n \alpha])! } 
} = 
$$

Часть множителей сразу сокращаются, например, все $ e $ или $ \sqrt{2 \pi}  $
$$
% числитель
\frac{  n^{n} \sqrt{n} }
% знаменатель
{
[n \alpha]^{[n \alpha]} \sqrt{2 \pi [n \alpha]}  
(n - [n \alpha])^{n - [n \alpha]} \sqrt{ (n - [n \alpha])} 
} = 
$$

Функция "Целая часть" несколько усложняет сокращение.

Очевидно, $ \sqrt{n} $ или $ \sqrt{n \alpha} $ не являются существенными при оценке.

Наиболее существенными является множителиЖ
$ [n \alpha]^{[n \alpha]} $ и $ (n - [n \alpha])^{(n -[n \alpha])} $

Рассмотрим первый из них: 

Целая часть любого числа - это само число минус дробная часть.
$$
[k] = k - \epsilon
$$
? где $ \epsilon $ - дробная часть

Поэтому:
$$
[n \alpha]^{[n \alpha]} 
= (n \alpha - \epsilon)^{(n \alpha - \epsilon)}
= 
\underbrace{ (n \alpha)^{n \alpha - \epsilon}  }_{(n \alpha)^{n \alpha}}
\underbrace{(1 - \frac{\epsilon}{n \alpha})^{n \alpha - \epsilon}}_{(1 + o(1))^{n \alpha}} 
$$

Точная оценка эквивалентности этого выражения затруднена,
поэтому будем интересоваться его главной частью (см. примечания под выражением).
Чтобы применить эти выражения в оценке, небходимо поделить $[n \alpha]^{[n \alpha]} $ на какую-то функцию $ r_1(n) $, растущую не быстрее $ n $: $ 1 < r_1(n) < n $
$$
\frac{
(n \alpha)^{n \alpha} (1 + o(1))^{n \alpha}
}
{
r_1(n)
}
$$

Теперь рассмотрим выражение $ (n - [n \alpha])^{(n -[n \alpha])} $
К нему можно применить тот же прием - т.е. оценить главную часть.

Только в данном случае нужно не поделить, а умножить н какую-то функцию $ r_2(n) $, устроенную так же. как и $ r_1(n) $: 
$ 1 < r_2(n) < n $

Умножать нужно, т.к. левая часть равенства с показателем 
$ (n -[n \alpha]) $ больше превой с показателем $ (n - n \alpha)$.
Это логично, ведь вычитание в показателе числа без дробной части дает число больше, чем при вычитании с дробной 
($ (n - [3.5]) > (n - 3.5)  $).

$$
(n - [n \alpha])^{(n -[n \alpha])} = (n - n \alpha)^{(n -n \alpha)} (1 + o(1))
$$

Соберем все части оцененного выражения вместе:
$$
\frac{n^{n} \sqrt{n} r_3(n)}
{(n \alpha)^{n \alpha} n(1 - \alpha)^{n(1 - \alpha)}} \cdot
\frac{1}{(1 + o(1)) \sqrt{2 \pi (n \alpha)(n - n \alpha)}}
$$
Во второй дроби собраны части, незначительно влияющие на оценку, причем функции целой части убраны как незначащие.

В числителе присуствует функция $ r_3(n) $, такая что
$$
r_3(n) = \frac{r_1(n)}{r_2(n)}
$$
т.е. находится примерно между $ 1/n $ и $ n $

Теперь избавимся от множителей, незначительно влияющих на оценку (внесем их в некую функцию $ r_4(n) $):
$$
\sim \frac{
n^{n} r_4(n)
}{
(n \alpha)^{n \alpha} (n ( 1 - \alpha ))^{n(1-\alpha)}(1+o(1))^{n}
} =
$$

Функция $ r_4(n) $ устроена следующим образом:
$$
\frac{C_1}{n \sqrt{n}} < r_4(n) < C_2 \sqrt{n}
$$
где $ C_1, C_2 $ - некоторые константы

Преобразуем выражение дальше. Сократим все $ n $ (разность их показателей в числителе и знаменатале равна 0)

Также выражение $ (1 + o(1))^{n} $ близко к $ 1 $, поэтому его можно свободно перекидывать из знаменателя в числитель 
(а функцию $ r_4(n) $ можно внести под знак $ o(1) $).

$$
= 
\frac {
(1 + o(1))^{n} r_4(n) 
}
{ 
(\alpha^{\alpha} (1 - \alpha)^{(1 - \alpha)})^{n}
} 
= 
\frac {
(1 + o(1))^{n}
}
{ 
(\alpha^{\alpha} (1 - \alpha)^{(1 - \alpha)})^{n}
} =
\left( 
\frac{1}{\alpha^{\alpha} (1 - \alpha)^{(1 - \alpha)} } + o(1)
\right)^{n} =
$$

Введем следующее обозначение, называемое \textbf{энтропией}:
$H( \alpha) = 
- \alpha \ln{ \alpha } 
- (1 - \alpha) \ln({1 - \alpha})$

Тогда полученную оценку можно записать как
$$
C_{n}^{[n \alpha]} = (e^{H(\alpha)} + o(1) )^n
$$

Т.е., мы поняли, с какой скоростью в среднем растет подобный биномиальный коэффициент - 
за его рост отвечает функция $ H(\alpha) $

\textbf{Замечание}

\textbf{Возможна ли такая запись?}
$$
C_{n}^{[n \alpha]} 
= (e^{H(\alpha)} + o(1) )^n
= e^{n H(\alpha)}(1 + o(1) )
$$

Оценим $ (1 + o(1) )^{n} $ и попробуем найти контрпример.

Например, выражение
$$
\left( 1 + \frac{1}{\sqrt{n}} \right)^{n} 
= 
\left( 
\left( 1 + \frac{1}{\sqrt{n}} \right)^{ \sqrt{n} } 
\right)^{ \sqrt{n} } 
$$
может очень быстро расти.

%TODO вообще непонятно, ведь 1/sqrt(n) расходится?
\textbf{TODO вообще непонятно, ведь 1/sqrt(n) расходится?}

Т.е., число, ненамного большее 1 в очень большой степени может быть очень большим, и верно следующее утверждение

$$
(e^{H(\alpha)} + o(1) )^n 
\neq
e^{n H(\alpha)}(1 + o(1) )
$$

\subsection{Теорема. Асимптотики мультиномиального коэффициента}
$$
n = n_1 + n_2 + \cdots + n_k
$$
,$ k $ фиксированное число, причем $ n_i $ ведет себя следующим образом:
$$
n_i \sim \alpha_i n, \alpha_i > 0
$$

Тогда мультиномиальный коэффициент
$$
\binom{n}{n_1, n_2, \ldots, n_k} 
=\left(
e^{H(\alpha_1, \alpha_2, \ldots, \alpha_k)} + o(1)
\right)^{n}
$$

где функция $ H $ устроена следующим образом:
$$
H(\alpha_1, \alpha_2, \ldots, \alpha_k) = 
- \alpha_1 \ln \alpha_1 
- \alpha_2 \ln \alpha_2 
- \ldots
- \alpha_k \ln \alpha_k 
$$

Примем без доказательства (доказывается также по формуле Стирлинга (\ref{stirling}) )

\section{Оценка сумм последовательных биномиальных коэффициентов}

Рассматриваться будет только верхняя оценка.

Рассмотрим сумму последовательных биномиальных коэффициентов
$$
\sum_{k=0}^{m} C_n^k
$$

Докажем, что
$$
\sum\limits_{k=0}^{m} C_n^k 
\le
C_n^m \frac{n+1-m}{n+1-2m}
$$

С комбинаторной точки зрения это возможность выбрать из $ n $ элементов не более чем $ m $ элементов.

Также, очевидно
$$
m \le n/2
$$

Для начала, возьмем и поделим два последовательно идущих биномиальных коэффициента:
$$
\frac{
C_n^k
}
{C_n^{k-1}
}
=
% числитель
\frac{
(k-1)!(n-k+1)!
}
% знаменатель
{k!(n-k)!
}
= \frac{n-k+1}{k}
= \frac{n+1}{k} - 1
$$

Теперь, используя полученное выражение, поделим последовательные суммы биномиальных коэффициентов.

%TODO откуда следует? какие комбинаторные соображения
\textbf{TODO откуда следует? какие комбинаторные соображения}

$$
\sum\limits_{k=0}^{m} C_n^k 
= C_n^m \sum\limits_{k=0}^{m} \frac{C_n^k}{C_n^m} 
= C_n^m \sum\limits_{k=0}^{m} \frac{C_{n}^{m-k}}{C_n^m} 
$$

$$
C_n^m \sum\limits_{k=0}^{m} \frac{C_{n}^{m-k}}{C_n^m} = 
C_n^m
\left( 
1 +
\frac{1}{((n+1)/m - 1)} +
\frac{1}{((n+1)/m - 1)((n+1)/(m-1) - 1)} +
\frac{1}{((n+1)/m - 1)((n+1)/(m-1) - 1)((n+1)/(m-2) - 1)} + \ldots
\right) <
$$

Очевидно, каждый последующий знаменатель больше предыдущего, поэтому каждое следующее слагаемое меньше. Т.е. мы увеличим выражение, если будем писать только первое слагаемое.

$$
C_n^m
\left( 
1 +
\frac{1}{(n+1/m-1)} +
\frac{1}{(n+1/m-1)^{2}} +
\frac{1}{(n+1/m-1)^{3}} + \ldots
\right) < 
$$

Будем считать. что это бесконечная геометрическая 
прогресия (от этого сумма выражения только увеличится).
$$
C_n^m
\frac{1}{1 + \frac{1}{(n+1)/m - 1}}
$$

Упрощаем
$$
C_n^m 
\frac{\frac{n+1}{m} - 1
}
{\frac{n+1}{m} - 2
} = 
C_n^m \frac{n+1-m}{n+1-2m}
$$

Доказано

\textbf{Замечание}

Данная оценка становится грубой, если $ m $ 
приближается к $ \frac{n}{2} $

\subsection{Теорема. Энтропийная оценка}
$$
\sum\limits_{k=0}^{m} C_n^k \le e^{nH(\frac{m}{n})}
$$
$$
m \le n/2
$$

Доказательство

Возьмем уже известную нам сумму последовательных биномиальных коэффициентов и оценим ее сверху.
Для этого выберем некоторую $ t \le 1 $ 
$$
\sum\limits_{k=0}^{m} C_n^k \le \sum\limits_{k=0}^{m} C_n^k t^{k-m}
$$
Неравенство верно, т.к. для каждой $ t^{k-m} $
показатель степени $ (k-m)<0 $

Вынесем $ t^{-m} $ за знак суммы
$$
\sum\limits_{k=0}^{m} C_n^k t^{k-m}
= t^{-m} \sum\limits_{k=0}^{m} C_n^k t^{k} \le
$$

А выражение $ C_n^k t^{k} $ не что иное, как биномиальный коэффициент (если складывать не до
$ m $, а до $ n $, получаем бином Ньютона)

$$
\le t^{-m} (1 + t)^{n}
$$

$ t $ подберем так: $t = \frac{n}{n-m}$ 

$$
1+t = \frac{n}{n-m} 
$$

Тогда, раскрыв $ t $
$$
t^{-m} (1 + t)^{n} = 
% ()
\left( 
\frac{m}{n-m}
\right)^{-m}
% умножить на ()
\left( 
\frac{n}{n-m}
\right)^{n} = 
% =
\frac{ n^{n}
}{ m^{m} (n-m)^{n-m}
} =
$$
$$
% =
= \left(
\frac{1}{ \frac{m}{n}^{\frac{m}{n}} 
% *
(1-\frac{m}{n})^{1 - \frac{m}{n}} }
\right)^{n} =
e^{nH(\frac{m}{n})}
$$







