\section{Производная. Дифференцируемость функции. Дифференциал функции}

\subsection{Определение}

Производная (функции в точке) — основное понятие дифференциального исчисления, характеризующее скорость изменения функции (в данной точке). Определяется как предел отношения приращения функции к приращению ее аргумента при стремлении приращения аргумента к нулю, если такой предел существует. Функцию, имеющую конечную производную (в некоторой точке), называют дифференцируемой (в данной точке).


\subsection{Определение производной функции через предел}

Пусть в некоторой окрестности точки $x_0 \in R$ определена функция $f(x)$ Производной функции $f$ в точке $x_0$ называется предел, если он существует

$$
f'(x_0) = \lim\limits_{x \to x_0} \frac{f(x) - f(x_0)}{x - x_0} = \lim\limits_{\Delta x \to 0} \frac{f(x_0+\Delta x)-f(x_0)}{\Delta x} \eqno (1.1)
$$

\subsection{Геометрический и физический смысл производной}

\subsubsection{Геометрический смысл }

Если функция $f\colon U(x_0) \to R$ имеет конечную производную в точке $x_0,$ то в окрестности $U(x_0)$ её можно приблизить линейной функцией
: $f_l(x) \equiv f(x_0) + f'(x_0)(x-x_0)$

Функция $f_l$ называется касательной к $f$ в точке $x_0$. Число $ f'(x_0) $ является угловым коэффициентом или тангенсом угла наклона касательной прямой.

Т.е. касательная  - это предельное положение секущей.


Прямо из определения следует, что график касательной прямой проходит через точку $(x_0, f(x_0))$. Угол $\alpha$ между касательной к кривой и осью $Ox$ удовлетворяет уравнению:


$$\operatorname{tg}\,\alpha = f'(x_0)= k,$$


где ${tg}$ обозначает тангенс, а $\operatorname {k} $ — коэффициент наклона касательной.
Производная в точке $x_0$ равна угловому коэффициенту касательной к графику функции $y = f(x)$ в этой точке.

\subsubsection{Физический смысл}

Пусть $s=s(t)$ — закон прямолинейного движения. Тогда $v(t_0)=s'(t_0)$ выражает мгновенную скорость движения в момент времени $t_0.$ Вторая производная $a(t_0) = s''(t_0)$ выражает мгновенное ускорение в момент времени $t_0.$

Вообще производная функции $y=f(x)$ в точке $x_0$ выражает скорость изменения функции в точке $x_0$, то есть скорость протекания процесса, описанного зависимостью $y=f(x).$

\subsection{Уравнение касательной и нормали}

Уравнение произвольной прямой, проходящей через точку $ (x_{0}, y_{0}) $:
$ y - y_{0} = k(x - x_{0}) $, где $ k $ - угловой коэффициент.

А т.к. мы ищем касательную, то он равен производной.

Таким образом, уравнение касательной в точке $ (x_{0}, y_{0}) $:

$$
f(x) = f(x_0) + f'(x_0)(x-x_0)
$$

Точно так же находится нормаль - уравнение прямой, перпендикулярно данной, получается следующей заменой углового коэффициента: $ k_{2} = -\dfrac{1}{k_{1}} $

Таким образом, заменив в уравнении (1) угловой коэффициент, мы получим уравнение нормали:

$$
f(x) = f(x_0) - \dfrac{1}{f'(x_0)} (x-x_0)
$$


\subsection{Производная справа и слева}

Правая производная

$ f(x+0) = \lim\limits_{\Delta x \to 0+0} \dfrac{\Delta x}{\Delta y}
 = \lim\limits_{\Delta x \to 0, \Delta x > 0} \dfrac{\Delta x}{\Delta y} $

Левая производная

$ f(x-0) = \lim\limits_{\Delta x \to 0-0} \dfrac{\Delta x}{\Delta y}
 = \lim\limits_{\Delta x \to 0, \Delta x < 0} \dfrac{\Delta x}{\Delta y} $

Если правая и левая производная равны, то производная в точке существует.


\subsection{Бесконечная производная}

Производные называются бесконечными, если предел в точке равен $ +/- \infty $.

$ f'(x) = \lim\limits_{\Delta x \to 0} \dfrac{\Delta x}{\Delta y} = + \infty $

$ f'(x) = \lim\limits_{\Delta x \to 0} \dfrac{\Delta x}{\Delta y} = - \infty $

Очевидно, касательная в этом случае вертикальна ($ y = c, c $ - константа).

Если $ f'(x) = + \infty $, функция возрастает, иначе - убывает.

Пример: $ f(x) = \sqrt[3]{x}$

$ f'(x) = \dfrac{1}{\sqrt[3]{x^{2}}}$

$f'(0) = \infty $

Считается, что в таких точках функция не дифференцируема.

\subsection{Гладкая функция. Угловая точка}

Гладкая функция - имеющая касательную в каждой точке.

Угловая точка - случай, когда производные слева и справа не равны друг другу.

Пример: $ f(x) = |x| $

$ f(x+0) = 1 $

$ f(x-0) = -1 $

Производная в угловой точке не существует.

\subsection{Дифференцируемая функция. Дифференциал}

Функция $f\colon M\subset R \mapsto R$ одной переменной является дифференцируемой в точке $x_0$ своей области определения $M$, если существует такая константа $a$, что для любой точки $x \in M$ верно

$f(x)=f(x_0) + a(x-x_0) + o(x-x_0), \ \quad x \to x_0,
$

при этом число $a$ неизбежно равно производной 

$a = f'(x_0) = \lim \limits_{x\to x_0}\frac{f(x)-f(x_0)}{x-x_0}.$

Функция одной переменной является дифференцируемой в точке $x_0$ тогда и только тогда, она имеет производную в этой точке.

Если функция дифференцируема на интервале $(a,b)$, то она непрерывна на интервале $(a,b)$. Обратное, вообще говоря, неверно (например, функция $y(x)=|x|$ на $[-1,1]$)

Примеры: 

Функция Дирихле, разрывна в каждой точке.

$ D(x) = \begin{cases}1, &      x\in \mathbb Q, \\
 0, & x \in \mathbb R \backslash \mathbb Q. \end{cases} $
 
Функция Вейерштрасса — пример непрерывной функции, нигде не имеющей производной.

$ w(x)= \sum_{n=0}^\infty b^n \cos(a^n \pi x) $

Функция имеет фрактальную структуру, поэтому каждая точка является угловой.


