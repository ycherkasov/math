\section{Понятие обратной функции. Производная обратной функции. Производные гиперболических функций. Логарифмическое дифференцирование. Применение дифференциалов в приближённых вычислениях}

\subsection{Понятие обратной функции. Теорема о производной обратной функции}

\subsubsection{Обратная функция. Взаимно обратные функции. Примеры}

Функция $ \phi $ является обратной к функции $ y $, если для каждого $ y = f(x) $ существует $ x = \phi(y) $.
Функции $ \phi $ и $ y $ называются взаимно обратными.

Чтобы найти обратную функцию, нужно решить уравнение $x = F(y)$ относительно $y$. Если оно имеет более чем один корень, то функции обратной к $F$ не существует. Таким образом, функция $f(x)$ обратима на интервале $(a;b)$ тогда и только тогда, когда на этом интервале она инъективна (одному значению соответствует единственное).

График обратной функции симметричен относительно прямой $ y = x $. Обратна яфункция сохраняет свойство возрастания-убывания.

Примеры:

\begin{itemize}
\item
Если $F:\mathbb{R} \to \mathbb{R}_+,\; F(x) = a^x$, где $a>0,$ то $F^{-1}(x) = \log_a x$

\item
Если $F(x) = ax+b, \; x\in \mathbb{R}$, где $a,b\in \mathbb{R}$ фиксированные постоянные и $a \neq 0$, то $F^{-1}(x) = \frac{x-b}{a}.$

\item
Если $F(x)=x^n,x \ge 0, n\in \mathbb Z$, то $F^{-1}(x)=\sqrt [n] {x}.$
\end{itemize}


\subsubsection{Теорема о существовании обратной функции}

Для непрерывной функции $F(y)$ выразить $y$ из уравнения $x - F(y) = 0$ возможно в том и только том случае, когда функция $F(y)$ монотонна

\subsubsection{Теорема о производной обратной функции}

Для дифференцируемой функции с производной, отличной от нуля, производная обратной функции равна обратной величине производной данной функции, т.е
$y'_x=\frac{1}{x'_y}$

Доказательство элементарно из определения производной и обратной функции.


\subsection{Производные обратных тригонометрических функций}

\subsubsection{Арксинус}

$ y = \arcsin x, x = \sin y $

Тогда

$ x'_y = \sin 'y  = \cos y, x \in [-1, 1] $

По формуле производной обратной функции

$ y'_{x} = \dfrac{1}{\cos y} = \dfrac{1}{ \sqrt{1 - \sin^{2} y} } = \dfrac{1}{ \sqrt{1 - x^{2}} } $

$ \arcsin'{x} = \dfrac{1}{ \sqrt{1 - x^{2}} } $

\subsubsection{Арктангенс}

$ y = \arctg x, x = \tg y $

Тогда

$ x'_y = \tg 'y = \dfrac{1}{\cos^{2}y}, y \in (-\dfrac{\pi}{2};\dfrac{\pi}{2}) $

$ y'_{x} = \dfrac{1}{\frac{1}{cos^{2}y}} = \dfrac{1}{1+\tg^{2} y} = \dfrac{1}{1+x^{2}} $

\subsection{Логарифмическое дифференцирование}

Логарифмическим дифференцированием называется метод дифференцирования функций, при котором сначала находится логарифм функции, а затем вычисляется производная от него. Такой прием позволяет эффективно вычислять производные функций, которые дифференцируются сложнее обычным методом.

$ y = f(x) $

$ \ln ' y = \dfrac{y'}{y} $ (как производная сложной функции)

Следовательно, $ y' = y(\ln' y) $

В основном это степенные и рациональные функции общего вида $ f(x)^{u(x)} $ или $ \dfrac{f(x)^{n}}{u(x)^{m}} $ 

Найдем формулу дифференцирования функций вида $ f(x)^{u(x)} $

$ \ln y = \ln (f(x)^{u(x)}) = u(x) \ln f(x)  $

$ \dfrac{y'}{y} = u'(x) \ln f(x) + u(x) \dfrac{f(x)}{f'(x)} $

Формула логарифмического дифференцирования

$$
y' = f(x)^{u(x)} \left( u'(x) \ln f(x) + u(x) \dfrac{f(x)}{f'(x)} \right)  
$$

\subsection{Пример 1}

$ y = x^{x} $

$ \ln y = x \ln x $

$ \dfrac{y'}{y} = (x \ln x)' = \ln x + 1 $

Продифференцируем

$ y' = x^{x}(\ln x + 1) $

\subsection{Пример 2}

$ y = \dfrac{(x+1)^{2}}{(x+2)^{3}(x+3)^{4}} $

$ \ln y = \ln{(x+1)^{2}} - \ln {(x+2)^{3}} - \ln {(x+3)^{4}} $

$ \ln y = 2 \ln{(x+1)} - 3 \ln {(x+2)} - 4 \ln {(x+3)} $

Продифференцируем

$ \ln ' y = \dfrac{y'}{y} = \dfrac{2}{(x+1)} - \dfrac{3}{(x+2)} - \dfrac{4}{(x+3)} $

$ y ' = \dfrac{(x+1)^{2}}{(x+2)^{3}(x+3)^{4}}$
$\left(  \dfrac{2}{(x+1)} - \dfrac{3}{(x+2)} - \dfrac{4}{(x+3)} \right)  $

\subsection{Приближенные вычисления}

$ f(x_{0}+\Delta x)\approx f(x_{0}) + f'(x_{0}) \Delta x $ 

(доказательство из определения дифференциала и $ \Delta y $)

\subsection{Пример}

$ y = \sqrt{x} $

$ y = \sqrt{x_{0} + \Delta x} \approx \sqrt{x_{0}} + \dfrac{1}{2 \sqrt{x_{0}}} \Delta x $

(из определения производной степенной функции)

$ \sqrt{3,996} = \sqrt{4 - 0,004} \approx 2 + \dfrac{1}{2 \sqrt{4} }(-0,004) = 1,999 $
