\chapter{ Лекция 1 }
\section{Основы комбинаторики.}
\subsection{Основные правила}

 
Пусть $A, B$ --- множества объектов. В $A$ ровно $n$ объектов, в $B$ ровно $m$ объектов.

 $A =  \{ a_1, \dots, a_n \} $ 
 
 $B =  \{ b_1, \dots, b_m \} $
 
\begin{description}
\item[Правило сложения]~ 

Если элемент из множества $A$ можно выбрать $n$ способами, а элемент из множества $B$ можно выбрать m способами, то выбрать элемент из  $A$ или из $B$ можно $n+m$ способами.
\item[Правило умножения]  ~

Если мы берём сначала элемент из $A$, затем элемент из $B$, то количество способов $m \times n $ .
\item[Принцип Дирихле] ~

Если $n+1$ кроликов сидят в $n$ клетках, в одной клетке точно есть по крайней мере два кролика. 
\end{description}

\subsection{Сочетания, размещения, перестановки}

Пусть $A$ --- множество объектов. 

$A =  \{ a_1, \dots, a_n \}$

Зафиксируем $k \in \naturalset $. Подмножество $A$ мощности $k$ называется \textit{ $k$-сочетанием (без повторений) } . 

Если учитывать порядок элементов, то говорим уже о размещениях.

Есть также сочетания с повторениями, когда мы можем брать совпадающие элементы, но всё-таки сваливать их в кучу. Их порядок по-прежнему может быть любым.

Размещения с повторениями --- такой набор элементов из множества $A$, где могут встречаться повторяющиеся элементы, однако порядок внутри этого набора нам важен. 

$C^k_n$ --- количество $k$-сочетаний без повторений.

$\bar{C}^k_n$ --- количество $k$-сочетаний с повторениями.

$A^k_n$ --- количество $k$-размещений без повторений.

$\bar{A}^k_n$ --- количество $k$-сочетаний без повторений.


По правилу умножения

\[ \bar{A}^k_n  = n^k .\]

Аналогично, 

\[ A^k_n = n(n-1) ... (n-k+1) = \frac {n!} {(n-k)!} .\]

Следствие:

\[ C^k_n = \frac{a^k_n} {k!} = \frac {n!} {(n-k)! k!} .\]

\begin{description}
\item[Перестановка] --- один из способов переставить \ элементы множества в различном порядке. Другими словами, это биекция множества на себя. Вполне очевидно, что количество перестановок выражается, как

$ A^n_n = n!$.

\end{description}

\begin{alem} Имеет место равенство:

$ \bar{C}^k_n = C^k_{n+k-1}$

\end{alem}

\begin{proof} 

Рассмотрим $ A = \{a_1, ..., a_n\} $. Извлекаем "кучки" \  в сумме $k$ объектов. 
Сопоставим $k$-сочетанию битовую строку:

$
\underbrace {11} _{\text{ количество } a_1} 0 
\underbrace {111} _{\text{ количество } a_2} 0...011$ --- в коде (строке) всего $ k+n-1 $ символов.

Если количество $a_i$ равно нулю, то ставим ноль вместо соответствующих ему единиц. По последовательности длины $n+k-1$ c $k$ нулями можно восстановить $k$-сочетание с повторениями. 
\end{proof}
 
\section{Задача о перестановке букв в слове}
Дано слово из $N$ букв. Оно состоит из символов $a_1, \dots, a_n$, каждый из которых встречается $\alpha_i$ раз. Нам необходимо посчитать количество слов, которое можно составить из этих букв. Естественно, что так как буквы могут повторяться, то количество слов может оказаться меньше, чем количество перестановок из букв.
 
Мы можем обобщить задачу. Пусть у нас есть $n$ объектов, из них $n_1$ объектов первого типа, $n_2$ объектов второго типа, и т.л. Объектов $k$-ого типа --- $n_k$, Сколько существует различных последовательностей  из этих объектов?

\begin{thm} Это количество выражается следующей формулой:

$ P(n_1, \dots, n_k) = \frac{n!}{n_1 ! n_2! \dots n_k !} $
\end{thm}

\begin{proof}
Всего есть $n= \sum \limits _i n_i$ позиций, на которые мы можем размещать объекты. Способов выбрать позиции для $a_1$ у нас $C^{n_1}_n$. Как только эти позиции зафиксированны, объекты первого типа уже расставлены, у нас осталось $n-n_1$ позиций. Таким образом, способов разместить $a_2$ --- $C^{n_2} _{n-n_1}$. Мы перемножаем все эти способы, последовательно размещая объекты.

$\displaystyle 
C^{n_1}_n \times C^{n_2}_{n-n_1} \times \dots \times C^{n_k}_{n-n_1-\dots-n_{k-1}}= \\
= \frac {n!} {n_1! (n-n_1)!} \frac {(n-n_1)!} {n_2! (n- n_1 - n_2)!} \times \dots \times \frac {(n-n_1 - \dots - n_{k-1})!} {n_k! 
  \underbrace{(n-n_1 - \dots - n_k)!} _1 } = \\
= \frac {n!} {n_1! n_2! \dots n_k!}
$
\end{proof}





\section{Бином Ньютона и полиномиальная формула}

\subsection{Бином Ньютона}

\[ (x+y)^n=\sum \limits_{k=0}^n C^k_n x^k y^{n-k}  \] 

$C^k_n$ называются биномиальными коэффициентами.

\subsection{Полиномиальная формула} 
\[\displaystyle
(a_1 + a_2 + \dots + a_k)^n = \sum \limits _{n_1+\dots + n_k = n} \frac {n!} {n_1! \dots n_k!} a_1^{n_1} a_2^{n_2} \dots a_k^{n_k} 
\]
\begin{proof}   
TODO Число способов выбрать x с коэффициентом 

Расписываем $ \sum\limits_{i=0}^k x_k ^n $, получаем требуемое. 
\end{proof}

В связи с последней формулой числа $ P(n_1,...,n_k) $ называются полиномиальными коэффициентами .
Иными словами, это количество способов, которыми можно составить слово из букв $a_i$, каждая из которых встречается $n_i$ раз, а всего в слове $n $ букв.

\[ (x_1 + ... + x_k)^n = \sum _{\substack{
   \forall n_i : n_i\geq 0; \\
 n_1 + ... + n_k = n
    } }
 P(n_1, n_2, ...,n_k)x_1^{n_1}x_2^{n_2}...x_k^{n_k}
\]

\section{Тождества}
\begin{enumerate}
 


\item $ C ^k _n = C^{n-k}_n $

\item $ C^ k _n = C^k _{n-1} + C^{k-1} _{n-1}$

\begin{proof} Комбинаторно можно рассуждать так. Пусть у нас есть множество объектов

$A = \{a_1, a_2,  \dots, a_n\}$.

Тогда $ C^k_n $ --- количество k-сочетаний из этого множества объектов, а $ C^k_{n-1} $ --- количество $k$-сочетаний из множества $A'$:

$A' = \{a_2,  \dots, a_n\} = A \backslash \{a_1\} $

$ C^{k-1}_{n-1} $ --- это количество  $k-1$-сочетаний из $A'$. Но любое такое сочетание, будучи дополненным символом $a_1$, даст $k$-сочетание!

Другими словами, любое количество сочетаний $C^k_n$ складывается из числа сочетаний, не содержащих $a_1$ (коих   $ C^k_{n-1} $  ), и из сочетаний, содержащих $a_1$ (количество сочетаний, обязательно содержащих $a_1$ ---  $ C^{k-1}_{n-1} $ ).
\end{proof}

\item 

$C^0_n + C^1_n + \dots + C^n_n = 2^n$
\begin{proof}[Доказательство 1] 
Представим как бином:

$(1+1)^n = 2^n = \sum \limits _{i=0} ^n {C^ i _n} $
\end{proof}
\begin{proof}[Доказательство 2] 
Рассмотрим множество $n$-ок (кортежей с фиксированным числом элементов), состоящих из нулей и единиц.

$ A = \{ (0,\dots, 1,0, \dots, 1 )\} $

Тогда, очевидно, мощность такого множества $2^n$. Посчитаем её другим способом. Найдём количество $n$-ок, в которых ровно $k$ единиц. Очевидно, их будет $C^k_n$. Просуммировав их, получим доказываемое тождество.
\end{proof}

\item $ \sum  _{\substack{
     (n_1, ..., n_k): \\   
      n_i \geq 0, \sum n_i = n }}
    P(n_1, ..., n_k) = k^n$ 
\begin{proof}
Подставим вместо иксов в полиномиальную формулу единицы. 
\end{proof}

\item $\sum \limits _{i=0} ^n {(C^ i _n )^2 } = C^n _{2 n}$
\begin{proof}
Рассмотрим множество :
\[
A = \{a_1, ..., a_n, a_{n+1}, ... ,a_{2n} \} , \quad |A| = 2n .
.\]

Сколько существует сочетаний без повторений?
Взяв $k$ элементов из первой половины, из второй мы возьмём $n-k$.

\[ C^ n _{2 n} = \sum \limits_{k=0}^n C^k _n C^{n-k} _n = \sum \limits _{k=0} ^n (C^k _n) ^2 \]
\end{proof}


\item Выведем формулу для суммы степеней натуральных чисел от 1 до $k$.

Рассмотрим множество мощности $n+1$:

$ \{a_1, \dots, a_{n+1} \}, \quad m \in \naturalset $ 

Всего в нём столько $m$-сочетаний с повторениями: 

\[ C^m _{(n+1)+m-1} = C^m_{n+m} \]

Разобьём эти сочетания на несколько классов:

\begin{enumerate}

\item[1] Все те, которые не содержат объекта $a_1$. Их $C^m_{n+m-1}$

\item[2] Все те, которые содержат ровно один объект $a_1$. Их $C^{m-1}_{n+m-2}$

\dots

\item[m] Все те, которые содержат ровно $m$ объектов $a_1$
\end{enumerate}

Итак, предварительный итог нашей работы:

\[
\begin{aligned}
C^m_{n+m} = &C^m_{n+m-1} + C^{m-1}_{n+m-2} + \dots = \\
&C^{n-1}_{n+m-1} + C^{n-1}_{n+m-2} + \dots 
\end{aligned}
.\]

Теперь рассмотрим разные случаи:

\begin{itemize}
\item 
$n= 1 \Rightarrow C^1_{m+1} = \underbrace{1+1+\dots+1}_{m +1 \text{ штук}}$

\item 
$
n= 2 \Rightarrow 
C^2_{m+2} = C^1_{m+1} + C^1_{m} + C^1_{m-1}+ \dots = 
1 + 2 + 3 + \dots + m + (m+1)
$ 

Это арифметическая прогрессия. Cлева мы имеем

$\displaystyle C^2_{m+2} = \frac {(m+1)!} {2! m!} = \frac {(m+2) (m+1)} {2}$.

Таким образом, мы вывели формулу суммы арифметической прогрессии.

\item 
$
n= 3 \Rightarrow 
C^3_{m+3} = C^2_{m+2} + C^2 _{m+1} + \dots
$

$\displaystyle  \frac {(m+1) (m+2) (m+3) } 6 = \frac {(m+1)(m+2)} 2 + \frac {m(m+1)} 2 + \frac {m(m-1)} 2 + \dots = $

$=\displaystyle  \frac {(m+1)^2} 2 + \frac {(m+1)} 2 + \frac {m^2} 2 +  \frac {(m-1)^2} 2 +  \frac {(m-1)} 2 + \dots = $

 
 $\displaystyle =\frac 1 2 \bigg(1^2 + 2^2 + \dots + (m+1)^2 \bigg) + \frac 1 2 \bigg( \underbrace{ 1 + 2 + \dots + (m+1)} _{\frac {(m+1)(m+2) } 2} \bigg) $


$\displaystyle 1^2 + 2^2 + \dots + (m+1)^2 = 2 \bigg( \frac {(m+1) (m+2) (m+3) } 6 - \frac {(m+1)(m+2)} 4 \bigg) = $

$\displaystyle = (m+1)(m+2) \bigg( \frac {m+3} 3 - \frac 1 2  \bigg) $

Окончательно имеем

\[\displaystyle  1^2 + 2^2 + \dots + k^2  = \frac {k(k+1)(2k+1)} 6 .\]

\item $n \geq 4$ \quad Аналогичные рассуждения помогут нам найти формулу для суммы кубов и т.п.



\end{itemize}


\begin{rem}
Натуральные числа включают 0
\end{rem}

  
\item 
$ C^0_n - C^1_n + C^2_n - \dots + (-1)^n C^n_n = 
\begin{cases} 
0, \quad n \geq 1\\
1, \quad n = 0
\end{cases}
$
\end{enumerate}
 
TODO доказательство $(1-1)^n$

\begin{incexc} 
Даны множества $ S_1, S_2, \dots , S_n $
\[ |S_1 \cup S_2 \cup \dots \cup S_n| = |S_1| + |S_2| + \dots + |S_n| - |S_1 \cap S_2| - \dots + |S_1 \cap S_2 \cap S_3| - \dots + (-1)^{n+1} |S_1 \cap \dots \cap S_n|
\]
--- "интуитивно понятная формула". 
\end{incexc}

Теперь рассмотрим объекты $ a_1, \dots, a_N$ и свойства $ \alpha _1, \alpha _2, \dots, \alpha _M$ 

$ N(\alpha _i , \alpha _j, \alpha _k, \dots) $ --- количество объектов из исходного множества, которые обладают перечисленными в скобках свойствами. Не обязательно только ими!

$ \alpha ' _i $ --- свойство "не обладать свойством  $ \alpha _i $"

Другой вариант формулы включений и исключений имеет вид:

$N(\alpha'_1, \dots, \alpha'_M) = N - N(\alpha_1) - N(\alpha_2) - \dots - N(\alpha_1) + \\
+ N(\alpha_1,\alpha_2) + N(\alpha_1,\alpha_3) + \dots + N(\alpha_{N-1},\alpha_N) + \dots + (-1)^M N(\alpha_1,\dots, \alpha_M)  $

TODO есть множество из m объектов, m-размещения с повторениями. Свойств n штук.

\begin{cyclestask}

Дан алфавит  $ X = \{b_1, \dots, b_n \}$. Будем составлять слова длины $n$ из его символов:

$ a_1 \quad  a_2  \quad \dots  \quad a_n , \quad a_i \in\chi $

\setlength{\unitlength}{1mm}
\begin{picture}(40,10)
\put(0,7){\vector(1,0){5}}
\put(0,7){\circle*{1}}
\put(8,7){\vector(1,0){5}} 
\put(8,7){\circle*{1}}
\put(26,7){\vector(1,0){5}}
\put(26,7){\circle*{1}}
\end{picture}

Если мы хотим подчеркнуть тот факт, что наша последовательность обычная, линейная, незацикленная, будем рисовать точки вместо символов и соединять их направленными стрелками. Количество таких "обычных линейных последовательностей" \ ---  $r^n$.
 
 Для краткости так будем обозначать линейное слово следующим образом:
 
$a_1 a_2 \dots a_n 
$.


Пусть дано линейное слово $a_1 a_2 \dots a_n $. Назовём циклическим сдвигом преобразование:
\[
a_1 a_2 \dots a_n  \mapsto 
a_2 a_3 \dots a_n a_1
\]

\newpage  %TODO тут надо просто сделать неразрывный блок, иначе рисунок переносится на новую страницу.

Что будет, если мы зациклим последовательность? Добавим еще одну стрелку, которая идёт из $a_n$ в $a_1$.

$ a_1 \quad  a_2  \quad \dots  \quad a_n , \quad a_i \in\chi $

\setlength{\unitlength}{1mm}
\begin{picture}(40,20)
\put(0,17){\vector(1,0){5}}
\put(0,17){\circle*{1}}
\put(8,17){\vector(1,0){5}} 
\put(8,17){\circle*{1}}
\put(26,17){\vector(1,0){5}}
\put(26,17){\circle*{1}}
\put(31,17){\circle*{1}}
\qbezier(31,17)(15,2)(4,13)
\put(4,13){\vector(-1,1){4}}
\end{picture}
  
Такая последовательность называется циклической. 

 $ (a_1 a_2 \dots a_n ) $ --- циклическое слово.

Нас интересует $T_r(n)$ --- число всех принципиально различных циклических последовательностей. 

\begin{example}
Возьмём алфавит $\chi = \{C,H,O\}$. Составим слово и зациклим его:

\[
 \xymatrix{ C \ar[r] & O \ar[d] \\
               O \ar[u] & H \ar[l] }
\]

Сколько различных линейных слов отвечает данному циклическому? Очевидно, что четыре. 
\end{example}

\begin{example}
Сколько различных линейных слов отвечает данному циклическому?
\[
 \xymatrix{ C \ar[r] & O \ar[d] \\
               O \ar[u] & C \ar[l] }
\]

Здесь ответ, очевидно, два.

$COCO \mapsto  OCOC  \mapsto  COCO $
\end{example}

Чтобы вывести формулу для $T_r(n)$, нам потребуются некоторые дополнительные выкладки.

\end{cyclestask}

 
\section{Формула обращения Мёбиуса}

 \begin{description}
 \item[Простое число] --- натуральное число, большее единицы, которое делится только на самое себя и на единицу. 
 \end{description}
  
\begin{mainarithm}
Для любого числа $ n \in \naturalset, n > 1$ верно: 

$ n = p_1 ^ {\alpha_1} \dots p_S ^ {\alpha_S} $, где $p_1,\dots, p_S$ --- простые числа, а $\alpha_1,\dots, \alpha_S \in \naturalset$ 
\end{mainarithm}


\subsection{Функция Мёбиуса}
 
 Введём следующую функцию, называемую функцией Мёбиуса:
 
\[ 
\mu (n) = 
\begin{cases} 
	1, 
		\quad n = 1\\
	0, 
		\quad n = p_1^{\alpha_1} p_2^{\alpha_2} \dots p_k^{\alpha_k}, 		
		\quad \text{если } \exists i : \alpha_i \geq 2\\
	(-1)^S,
		 \quad n = p_1 \dots p_S .
\end{cases}
\]

Примеры:

$\mu (7) = -1 $

$\mu (10) = 1 $

$\mu (12) = 0 $
 
\begin{alem}

\[
\displaystyle
\sum \limits _{d|n} \mu(d) = 
\begin{cases} 
1, \quad n = 1 \\
0, \quad n > 1
\end{cases}
\] 
$  \quad d|n $- "$d$ является делителем $n$".


\end{alem}



\begin{proof}
Рассмотрим случаи:
\begin{itemize}

\item 
$
n = 1, \quad \sum\limits_{d|n} \mu(d) = \mu(1) = 1
$
\item
TODO развернуто более
$
n > 1 \Rightarrow    n = p_1 ^ {\alpha_1} \dots p_S ^ {\alpha_S} $

$
 d = p_1^{\beta_1} p_2^{\beta_2} \dots p_s^{\beta_s}, 0 \leq \beta_i \leq \alpha_i
$

Если 
$
\exists i : \quad \beta_i > 1 \Rightarrow \mu(d) = 0
$


$\displaystyle
\sum \limits _{d|n} \mu(d) =  
\sum\limits_{\beta_1 = 0}^1 \dots \sum \limits _{\beta_S = 0} ^1 \mu (p_1^{\beta_1}\dots p_S^{\beta_S} ) = \\
\sum \limits ^S _{k=0} \sum _{\substack {
( \beta_1, \dots, \beta_S) : \\
\beta_1 + \dots + \beta_S = k
}} (-1)^k = \sum \limits _{k=0} ^S C^k_S (-1)^k = 0 
$
\end{itemize}
\end{proof}
\subsection{Формула обращения Мёбиуса}

Пусть $ f,g: \naturalset \mapsto \naturalset,  f(n) = \sum \limits _ {d | n} g(d). $ Тогда 
\[
g(n) = \sum \limits _{d|n} \mu(d) f(\frac n d )
\]

\begin{proof}
$ \displaystyle
\sum \limits _{d|n} \mu(d) \sum \limits _{d'| ^n/_d} g(d') = \\
\sum \limits _{d|n} g(d) \sum \limits _{d'| ^n/_d} \mu(d') = \dots \\
$

Почему можно менять суммы местами? Рассмотрим $ n = 12 $ 

$
\begin{aligned}
&d = 1 :  &d' \in \{1,2,3,4,6,12\} \\
&d = 2 :  &d' = \{1,2,3,6\} \\
&d = 3 :  &d' = \{1,2,4\} \\ 
&d = 4 :  &d' = \{1,2,3\} \\
&d = 6 :  &d' = \{1,2\} \\
&d = 12: &d' = \{1\} \\
\end{aligned}
$

$
\begin{aligned}
&\mu(1)(g(1) + g(2) + g(3) + g(4) + g(6) + g(12)) + \\
+&\mu(2)(g(1) + g(2) + g(3) + g(6)) + \\
+&\mu(3) (g(1) + g(2) + g(4)) \\
+&\mu(4) (g(1) + g(3)) +\\
+&\mu(6)(g(1)+g(2)) \\
+&\mu(12)g(1)
\end{aligned}
$

- перегруппируем - 

$ 
g(1)(\mu(1) + \mu(2) + \mu(3) + \mu(4) + \mu(6) +   \mu(12) ) +\\
g(2)(\mu(1) + \mu(2) + \mu(3) + \mu(6)) + \dots
$  --- уже тут видно, что мы можем поменять суммы местами.


$
\displaystyle
d<n \Rightarrow \frac n d > 1,  \sum \limits _{d' | ^n/ _d} \mu(d') = 0
$

$
\displaystyle 
\begin{aligned}
0 &= g(n)\bigg(\sum\limits_{d'|1}\mu(d') \bigg) + \sum _{\substack {d|n \\ d < n}}g(d)\sum\limits_{d'|^n/_d} \mu(d') = \\
&= g(n) + \underbrace{\sum _{\substack {d|n \\ d < n}}g(d)\sum\limits_{d'|^n/_d} \mu(d') } _{ 0 \text{ по лемме} } = g(n)
\end{aligned}
$, что и требовалось доказать.



\end{proof}
 

Вернёмся к задаче о циклических словах и выведем формулу для $T_r (n)$. 

$X = \{b_1, \dots, b_r\} $ --- алфавит


\begin{description}
\item[Период линейной последовательности] это минимальное число сдвигов, при котором она переходит в себя.
\end{description}

\begin{alem}
Период последовательности обязательно является делителем длины для любого линейного слова. 
\end{alem}
 
 

Пусть $V$ --- множество линейных слов длины $n$.

$d_1, \dots, d_S $ --- все возможные делители числа $n$.

$|V| = r^n $, очевидно.

\begin{rem}
$ \sqcup $ --- символ дизъюнктного (без пересечений) объединения
\end{rem}

Разобьём  $V$ на непересекающиеся множества:


$
V = V_1 \sqcup V_2 \sqcup  \dots \sqcup V_S , \quad \{d: d|n\} = \{d_1, \dots, d_S\}
$

$V_i$ --- множество слов длины n и периода $d_i$.



\begin{alem} Если слово имеет длину $n$ и период $d$, то оно выглядит так:

$ 
a_1 \dots a_d a_1  \dots a_d \dots a_1 \dots a_d
$

и имеет $^n /_d$ "блоков".
\end{alem}
\begin{proof}
Очевидно, так как первые $d$ символов должны "переехать" \ в следующие $d$.
\end{proof}

\begin{cor} 
Между множеством $V_i$ и множеством  $W_i$ линейных последовательностей длины и периода $d_i$ есть взаимно однозначное соответствие (биекция). Значит, $|V_i| = |W_i|$.
\end{cor}


Значит, 
$r^n = \sum \limits _{d|n} |W_i|$
 
   
Обозначим за $U_i$  множество всех различных циклических слов, которые получаются из слов множества $W_i$ добавлением "стрелочки". Тогда

$ d_i |U_i| = |W_i|  $

Положим $|U_i| = M(d_i)$

$ r^n = d_1M(d_1)+\dots + d_S M(d_S) = \sum \limits _{d|n} dM(d) $

Если мы обозначим $r^n = g(n), \quad dM(d) = f(d)$, то по формуле обращения Мёбиуса 

$
f(n) = \sum \limits _{d|n} \mu(d)g\bigg(\frac n d\bigg) $, где $f(n) = nM(n)$/

Здесь $\displaystyle g\bigg(\frac n d \bigg) =  r ^ {^n/_d}$


$\displaystyle
f(n) = n M(n) =  \sum \limits _{d|n} \mu(d) r ^ {^n/_d}
$

$M(n)$ выражает число циклических последовательностей, которые отвечают линейным последовательностям длины $n$ с периодом $n$ 

Число же всех циклических последовательностей выражается, как 

$\displaystyle
T_r(n) = \sum \limits _{d|n} M(d) = \sum \limits _{d|n} \frac 1 d \sum \limits _{d'|d} \mu(d')r^{^d/_{d'}} 
$.

%lection1 ends