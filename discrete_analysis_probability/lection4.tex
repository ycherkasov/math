
\chapter{Лекция 4}

\section{Основы теории графов}

\begin{description}
\item[Граф] это пара из двух множеств --- вершин и рёбер. $ G = (V,E)$. Для нас $ |V| < \infty $.

$E $  --- любой набор различных пар несовпадающих элементов $V$, то есть отсутствие кратных рёбер и петель. Также граф изначально подразумевается неориентированным. 

\item[Мультиграф] разрешает кратные рёбра.
\item[Псевдограф] разрешает петли.
\item[Орграф] задаёт ориентацию рёбер, т.е. $(x,y) \neq (y,x) $

\item[Маршрут] --- последовательность вида вершина --- ребро --- вершина --- ребро и т.п. 
\item[Простая цепь]  ---  маршрут, где все вершины различны (и все рёбра, автоматически, тоже).
\item[Простой цикл] --- маршрут, где все вершины, кроме первой и последней, различны.
\item[Эквивалентность вершин.] Две вершины называются эквивалентными, если между ними в графе существует простая цепь.
\item[Компоненты связности] --- в графах это классы эквивалентности.
\item[Степень вершины $\deg v$] --- количество рёбер, в которые она входит как элемент пары.

В случае с орграфом мы говорим уже о двух характеристиках:

\begin{description}

\item[indeg $v$] --- количество рёбер, входящих в $v$;

\item[outdeg $v$] --- количество рёбер, выходящих из $v$.

\end{description}

Очевидно, что сумма всех степеней вершин равна удвоенному числу рёбер.

$\sum \limits _{v \in V} \deg v = 2 |E|  $

Интересным и важным вопросом явлется распределение степеней у вершин различных графов.

\item[Изоморфизм графов] Два графа $G(V_1, E_1) $ и $H(V_2, E_2$ изоморфны, если

 $\exists \phi : V_1 \leftrightarrow V_2 \quad \& \quad \bigg ( (x,y) \in E_1 \Leftrightarrow  (\phi(x), \phi(y)) \in E_2 \bigg ) $

Имеется проблема существования полиномиального алгоритма, проверяющего два графа на изоморфизм.


\item[Дерево] --- связный граф, у которого нет циклов (ациклический).

У понятия дерева есть еще три равносильных определения:

\begin{itemize}

\item Граф, у которого между любыми двумя вершинами есть ровно один путь по рёбрам (простая цепь).
\item Связный граф, у которого рёбер на единицу меньше, чем вершин.
\item Ациклический граф, у которого рёбер на 1 меньше вершин. 

\end{itemize}

\item[Полный граф] --- граф, в котором между любыми двумя вершинами есть ребро. Обозначается как $K_n$, где $n$ --- число вершин.
\item[Планарность графа.] Говорят, что граф планарен, если его можно изобразить на плоскости без пересечения рёбер. Два полных графа внизу изоморфны и планарны, видно, что граф справа --- это копия левого, но изображённая без пересечений рёбер.  

\setlength{\unitlength}{0.5mm}
\begin{picture}(60,60)

\put(0,0){\line(0,1){50}}
\put(0,0){\line(1,0){50}}
\put(0,0){\line(1,1){50}}

\put(50,50){\line(0,-1){50}}
\put(50,50){\line(-1,0){50}}
\put(0,50){\line(1,-1){50}}
 
\put(0,0){\circle*{5}}
\put(0,50){\circle*{5}}
\put(50,0){\circle*{5}}
\put(50,50){\circle*{5}}

\put(100,0) {
\begin{picture}(400,60)

\put(25,25){\line(1,1){25}} 
\put(25,25){\line(1,-1){25}} 
\put(25,25){\line(-1,0){25}} 
\put(0,25){\line(2,1){50}} 
\put(0,25){\line(2,-1){50}} 
\put(50,50){\line(0,-1){50}} 

\put(25,25){\circle*{5}}
\put(0,25){\circle*{5}}
\put(50,0){\circle*{5}}
\put(50,50){\circle*{5}}
\end{picture}
}
\end{picture}

Полный же граф на 5 вершинах не планарен.

\begin{picture}(400,60)
%\multiput(0,0)(10,0){7}{\line(0,1){60}}\multiput(0,0)(0,10){7}{\line(1,0){60}} 

\put(0,20){\circle*{5}}
\put(10,0){\circle*{5}}
\put(30,0){\circle*{5}}
\put(40,20){\circle*{5}}
\put(20, 40){\circle*{5}}

\put(0,20){\line(1,1){20}}
\put(10,0){\line(-1,2){10}}
\put(10,0){\line(1,0){20}}
\put(0,20){\line(1,1){20}}
\put(30,0){\line(1,2){10}}
\put(40,20){\line(-1,1){20}}

\put(0,20){\line(1,0){40}} 
\put(0,20){\line(3,-2){30}} 
\put(40,20){\line(-3,-2){30}} 

\put(20,40){\line(1,-4){10}} 
\put(20,40){\line(-1,-4){10}} 
\end{picture}


\item[Двудольный граф] --- граф, множество вершин которого можно разбить на две части таким образом, что каждое ребро графа соединяет какую-то вершину из одной части с какой-то вершиной другой части, то есть не существует ребра, соединяющего две вершины из одной и той же части. Полный двудольный граф обозначается как $K_{a,b}$, где $a,b$ --- мощности соответствующих множеств вершин, на которые граф можно разбить.

\item[Подграф] $ G'(V',E')$ является подграфом для $G(V,E)$ если $ V' \subseteq V \& E' \subseteq E$. 
Обратите внимание, например, здесь на правой картинке представлен подграф, включающий вершину, которая в нём не соединена ни с какой другой!!


\begin{picture}(400,60)
\put(0,0){\circle*{5}}
\put(50,0){\circle*{5}}
\put(0,50){\circle*{5}}

\put(0,0){\line(1,0){50}}
\put(0,0){\line(0,1){50}}
\put(50,0){\line(-1,1){50}}

\put(100,0) {


\begin{picture}(400,60)
\put(0,0){\circle*{5}}
\put(50,0){\circle*{5}}
\put(0,50){\circle*{5}}
 
\put(50,0){\line(-1,1){50}}

\end{picture}

}
\end{picture}

\item[Индуцированный подграф] --- тот , в котором сохранены все рёбра на данном множестве вершин $E'$. Подграф справа --- не индуцированный. 

\item[Остовный подграф] --- подграф, где $E' = E$.

\item[Гомеоморфизм графов.] Графы гомеоморфны когда можно свести один к другому путём удаления вершин и "стягивания" рёбер, например:



\begin{picture}(400,60)

\put(0,0){\line(0,1){50}}
\put(0,0){\line(1,0){50}} 

\put(50,50){\line(0,-1){50}}
\put(50,50){\line(-1,0){50}} 
 
\put(0,0){\circle*{5}}
\put(0,50){\circle*{5}}
\put(50,0){\circle*{5}}
\put(50,50){\circle*{5}}

\put(100,0) {

\begin{picture}(60,60)

\put(0,0){\line(0,1){50}}
\put(0,0){\line(1,0){50}} 

\put(50,50){\line(1,-1){25}}
\put(50,0){\line(1,1){25}}
\put(50,50){\line(-1,0){50}} 
 
\put(0,0){\circle*{5}}
\put(0,50){\circle*{5}}
\put(50,0){\circle*{5}}
\put(50,50){\circle*{5}}
\put(75,25){\circle*{5}}

\end{picture}
}

\end{picture}

\item[Эйлеровость.] Граф называется Эйлеров если существует вершина, от которой можно выстроить цикл Эйлера, который покрывает все рёбра и проходит по каждому из них ровно один раз.
\end{description}

\begin{pantrkur}
Граф планарен тогда и только тогда, когда он не содержит подграфов, гомеоморфных $K_5 $ и $K_{3,3}$.
\end{pantrkur}

\begin{thm}
Связный граф (мультиграф) Эйлеров тогда и только тогда, когда степени всех его вершин чётны.
\end{thm}


\subsection{Подсчёт числа связных графов с данным количеством рёбер}

Пусть $C^k_n$ --- число связных графов с $n$ вершинами и $k$ рёбрами.

$C(n,k) = 0, k < n-2$ --- очевидный факт.

\begin{keli}
Число связных графов с $n$ вершинами и $k$ рёбрами:

\[T_n = n^{n-2}\]

\end{keli}
\begin{proof} (с помощью кодов Прюфера)

Мы докажем теорему если установим биекцию между множеством всех деревьев на $n$ вершинах и всеми последовательностями из $n-2$ элементов, где все элементы от $1$ до $n$ . Таких последовательностей $n^{n-2}$. Опишем процесс составления биекции.
 
Возьмём дерево $D$ на $n$ вершинах. У него есть висячие вершины (степени 1). 

Возьмём минимальную из них по номеру, $b_1$.  Тогда $a_1$ --- другой конец висячего ребра, а ребро $e_1 = ( a_1, b_1 )$. 
Добавляем в код $a_1$  и отрываем от дерева вершину. У нас получается опять дерево, так как связность мы утратить не могли (отрезанная вершина --- висячая). 

$D \backslash e_1 = D_1$

Проделаем аналогичную процедуру. 

$e_2 = ( a_2, b_2 )$. 


Повторяем $n-2$ раз пока не останется одно ребро. Так мы получим последовательность:


 
$a_1 a_2 .. a_{n-2}, \quad a_i \in \{1,2, \dots , n\}$

Мы показали, что разным деревьям соответствуют разные коды. Теперь докажем, что по последовательности можно восстановить дерево.

Дана последовательность:

$ a_1, a_2, \dots , a_{n-2} $

$ 1,2,\dots, n$

Очевидно, что не все натуральные числа от $1$ до $n$ найдут себе соответствие среди элементов последовательности, так как их больше. Возьмём самое малое из чисел, которое не нашло себе пары среди элементов последовательности. Назовём его $b_1$. Составим пару $e_1 = (a_1, b_1)$ и выкинем из последовательности $a_1$, а из чисел --- $b_1$. 
Действуя таким образом мы вычленим $n-2$ "рёбер", а потом из оставшихся двух чисел сформируем последнее "ребро". 



Чтобы "добить" \ биективность, надо доказать, что любому коду соответствует дерево. 
В графе, полученном из любого кода, всегда $n$ вершин и $n-1$ ребро. Осталось доказать по индукции, что он ацикличен, что достаточно просто. 

\end{proof} 

\begin{thm}
Количество лесов на $n$ вершинах с $r$ компонентами, в которых выделены $1, \dots, r$ принадлежности различным компонентам, равно $r n^{n-1-r}$ 
\end{thm}
  

Пусть $C(n,k) $ --- количество связных графов с n вершинами и k рёбрами. На данный момент мы знаем, что:

$C(n,k) = 0, k < n-1$

$C(n,n-1) = n^{n-2}$
 
 Чем будет $C(n,n)$? Мы добавим в дерево одно ребро и, таким образом, получим цикл, причем только один! Соответственно, $U_n = C(n,n)$ --- количество унициклических связных графов на $n$ вершинах.




\begin{thm}  
$ U_n \sim \sqrt{\frac \pi 8} n^{n-^1/_2} $
\end{thm}

\begin{proof}
Рассмотрим унициклический граф $ G $. Он, очевидно, выглядит как цикл, к вершинам которого могут быть прицеплены деревья. Если удалить рёбра цикла, то останется лес (некоторые деревья могут состоять только из одной вершины), причем число компонент связности в нём будет равно величине цикла. 

Мы хотим перебрать все унициклические графы на $n$ вершинах. Пусть $ 3 \leq r \leq n$ --- длина цикла. Тогда 

$C(n,n) = \sum \limits _{r=3} ^n \bigg (C^r_n \frac {(r-1)!} 2  \times r n ^{n-1-r} \bigg ) = \sum \limits _{r=3} ^n \frac {n (n-1) \dots (n-r+1)} {r!} \frac {(r-1)!} 2  r n ^{n-1-r}  $

$C(n,n) = \sum \limits _{r=3} ^n \frac {n (n-1) \dots (n-r+1)}  2   n ^{n-1-r}  $

$C(n,n) = \sum \limits _{r=3} ^n \frac 1 2    n ^{n-1-r} n^r \bigg( 1 - \frac 1 n \bigg)   \bigg( 1 - \frac 2 n \bigg) \dots \bigg( 1 - \frac {r-1} n \bigg) $

\[
U_n  = \frac 1 2 n ^ {n-1} \sum \limits _{r=3} ^n \prod \limits _{j=1} ^{r-1} \bigg ( 1 - \frac j n \bigg)
\] 

Теперь чтобы вывести асимптотику всего выражения нам надо вывести асимптотику суммы. Разобьём сумму на две части:

$\sum \limits _{r=3} ^n \prod \limits _{j=1} ^{r-1} \bigg ( 1 - \frac j n \bigg) =
 \sum \limits _{r=3} ^{\lfloor n^{^5/_9}\rfloor } \prod \limits _{j=1} ^{r-1} \bigg ( 1 - \frac j n \bigg) + 
 \sum \limits _{r = \lfloor n^{^5/_9}\rfloor + 1 } ^ n \prod \limits _{j=1} ^{r-1} \bigg ( 1 - \frac j n \bigg)
$

Отдельно упростим произведение:

$ \prod \limits _{j=1} ^{r-1} \bigg ( 1 - \frac j n \bigg) = \exp \bigg (\sum \limits _{j=1} ^{r-1} \ln \bigg ( 1 - \frac j n \bigg) \bigg ) =  $

Теперь можно представить каждый логарифм в более удобном виде с помощью ряда Тейлора:

$ =  \exp \bigg (\sum \limits _{j=1} ^{r-1} \bigg ( \frac j n + o \bigg(\frac {j^2} {n^2} \bigg) \bigg) \bigg ) 
= \exp \bigg ( - \frac {r (r-1)} {2n} + O\bigg ( \frac {r^3} {n^2} \bigg )\bigg )$

Рассмотрим первую из двух сумм.

$\sum \limits _{r=3} ^{\lfloor n^{^5/_9}\rfloor } \exp \bigg ( - \frac {r^2} {2n} - \frac r {2n} + o\bigg ( \frac {r^3} {n^2} \bigg )\bigg ) $

$r \leq n^{^5/_9} \Rightarrow \frac {r^3} {n^2}  \leq n ^ {\frac {15} {9} - 2}$, что на бесконечности стремится к нулю, как и $ \frac r {2n} $

$\sum \limits _{r=3} ^{\lfloor n^{^5/_9}\rfloor } \exp \bigg ( - \frac {r^2} {2n}  + o(1)\bigg )\bigg ) \sim
\sum \limits _{r=3} ^{\lfloor n^{^5/_9}\rfloor } e^ {- \frac {r^2} {2n} } \sim \int_0^{+\infty} e^{-^{r^2} / _{2n}}\,\mathrm{d}r   $


Интеграл можно взять до бесконечности, так как он сходится. 
Воспользуемся следующим фактом:

$\frac 1 {\sqrt{2 \pi}} \int _{-\infty} ^{+\infty} e^{-\frac {x^2} 2} \, \mathrm{d}x = 1$

Наш интеграл сводится к нему путём замены $x = \frac r {\sqrt{n}}$

$  \int_0^{+\infty} e^{-{x^2} / _2}\sqrt{n}\,\mathrm{d}x = \frac 1 2 \sqrt{2\pi} \sqrt{n} $

Мы получили асимптотику для первой суммы. Проведя  аналогичные рассуждения, оценим вторую, она асимптотически стремится к нулю.

Таким образом ответ:

 $ U_n \sim \sqrt {\frac \pi 8 } n^{n-^1/_2}$
\end{proof} 

 

\section{Обобщение формулы Кэли}

 У формулы Кэли есть естественное обобщение:

Пусть $C(n,n+k)$ - число всех связных графов с $n$ вершинами и $n+k$ рёбрами.

$\displaystyle 
k = -1 \Rightarrow   C(n,n-1) = T_n = n^{n-2} \\
k = 0  \Rightarrow   C(n,n) = U_n \sim \sqrt{\frac \pi 8} n^{n-0.5} 
$

Далее:

$\displaystyle C(n,n+1) \sim \frac 5 {24} n ^{n+1}$


Общий результат датируется 1993 г.  Полное реккурентное соотношение достаточно сложно:

$
C(n,n+k) \sim \gamma _k n^{n+ \frac {3k-1} 2} \bigg(1 + O\bigg(\frac {k^{^3/_2}} {\sqrt n} \bigg )\bigg)
$

$
\gamma_k = \frac {\sqrt \pi 3^k (k-1)}{2^ {\frac {5k-1} 2} \Gamma(k/2)} \delta_k
$

$
\delta_1 = \delta_2 = \frac 5 {36} 
$

$\delta_{k+1} = \delta_k + \sum \limits _{h=1} ^{k-1} \frac {\delta_h \delta_{k-h}} {(k+1)} ( C^h _k)^{-1}, \quad k \geq 2$
\subsection{Число независимости, кликовое и хроматические числа}

Для графа $ G = (V, E) $  :

\begin{description}
\item[Число независимости] $\alpha(G) = \max \{|W| : W \subseteq V, \forall x, y \in W : (x,y) \notin E\}$

\item[Кликовое число] $\omega(G) = \max \{|W| : W \subseteq V, \forall x, y \in W : (x,y) \in E\}$, мощность максимального полного подграфа (клики). 

\item[Хроматическое число] $\chi(G) = \min \{\chi : V = V_1  \sqcup \dots \sqcup V_{\chi}, \forall i \forall x,y \in V_i: (x,y) \notin E  \}$ --- минимальное количество цветов, в которое можно покрасить вершины графа, чтобы концы любого ребра были разного цвета.

Всегда выполняется :
 $\chi(G) \geq \max \{ \omega(G), \frac {|V|} {\alpha(G)} \} $
 
Ведь для любой клики нужно как минимум столько цветов, сколько в ней есть вершин; кроме того в каждой раскраске, дающей хроматическое число, каждый цвет представляет собой независимое множество, и даже если мы раскрасим все независимые множества в разные цвета, потребуется не меньше $  \frac {|V|} {\alpha(G)} \} $ цветов. 
 

Предположим, что $\chi(G) < \frac {|V|} {\alpha(G)}$ . Тогда есть цвет, в который покрашено более, чем $ \frac {|V|} { \frac {|V|} {\alpha(G)}} = \alpha(G) $ вершин. Тогда в этот цвет покрашено как минимум одно ребро полностью. 
\end{description}

\subsection{Гиперграфы}

\begin{description}
\item[Гиперграф] ---обобщённый вид графа, в котором каждым ребром могут соединяться не только две вершины, но и любые подмножества вершин. Формально говоря, это пара из двух множеств $H = (V, E), E \subseteq 2^V$. $H$ называется k-однородным, если $\forall e \in E : |e| = k$. Для $k=2$ получаем обычный граф.

К каждому гиперграфу можно привязать граф пересечений. 

$H=(V,E) \mapsto G(E,F)$, то есть рёбрам гиперграфа сопоставляем вершины графа. Что касается рёбер, $e_1 ,e_2 \in E$ образуют ребро в $F$ если их пересечение непусто.

\end{description}
 
\begin{example}

Рассмотрим гиперграф:

$H = ( \{ 1,2,3,4,5\}, \{ (1,2,3), (2,3), (3,4,5), (1,4,5), (2,5)\}$

Пересечения:

$G(\{v_1, v_2, v_3, v_4, v_5\}, \{(v_1,v_2), (v_1,v_3), (v_1,v_4), (v_1,v_5), (v_2, v_3),(v_2,v_5),(v_3,v_4), \\
(v_3,v_5), (v_4,v_5)\})$

\end{example}


Рассмотрим гиперграф $H = (V, C^k_v) $ --- k-однородный гиперграф, где присутствуют все возможные рёбра, т.е. полный k-однородный гиперграф. Построим граф пересечений $G$ для него. 

Число независимости $\alpha(G) $ --- максимальное количество k-элементных подмножеств множества $V$, которые попарно не пересекаются.
 
 \[ \alpha(G) = \bigg[ \frac n k \bigg] \]


Кликовое число $\omega (G)$ --- максимальное количество k-элементных подмножеств $V$, которые попарно пересекаются.

$\forall k : \omega(G) \geq C^{k-1} _{n-1}$
 
\begin{proof}

\end{proof}

\begin{rem} Перестановки.

Дано множество и операция на нём: $V = \{1,2, \dots, n\} , \sigma : V \mapsto V$. Операция $\sigma$ называется перестановкой, если это биекция множества на себя.

Всего существует $n!$ различных перестановок: $\{\sigma_1, \dots, \sigma_{n!} \} \ni e$
 
 \end{rem}
 
 
\begin{erdesh} 
 
\[ \omega(G) = \begin{cases}
C^k_n, 2k > n\\
C^{k-1}_{n-1}, k \leq \frac n 2 
\end{cases}\]

 
\end{erdesh}
\newcommand{\fset}{\mathbb{F}}
\begin{proof}

Фиксируем один элемент  в множестве V. В нём осталось $n-1$ элементов. Рассматриваем только те k-элементные подмножества, содержащие зафиксированный элемент. Помимо зафиксированного, в них $k-1$, и, естественно, они все пересекаются.
Первая часть теоремы доказана, остался случай 
Докажем второй.

Пусть $\fset = \{ F_1, \dots , F_s\}$ --- совокупность k-элементных подмножеств множества $V=\{1,2,\dots, n\}, 2k \leq n$. Любые два множества из неё пересекаются:

$\forall i,j: F_i \cap F_j \neq \emptyset$

Фактически, это клика в графе пересечений $G$. Нам нужно доказать:
\begin{enumerate}
\item $\exists \fset : s = C_{n-1}^{k-1}$

\item $\forall \fset: s \leq C^{k-1} _{n-1}$
\end{enumerate}

\begin{enumerate}
\item

Возьмём $F = \{ F \subset   V: |F| = k, 1 \in F \} $. Это все подмножества, которые содержат первый элемент , размера k. Разумеется, количество таких подмножеств это количество способов выбрать $k-1$ элементов из $n-1$ оставшихся элементов множества. Все эти подмножества пересекаются хотя бы в первом элементе множества. 

$|F| = C^{k-1} _{n-1}$

\item Доказательство этого пункта за авторством Katona.

Рассмотрим множество $\aset = \{A_1, \dots , A_n\}$, состоящее из множеств $A_i$.

$A_1 = \{1,2, \dots, k\}$

$A_2 = \{2, \dots, k, k+1\}$

$\dots$ , если мы "выскакиваем " \ за границу справа, то " вылезаем"  \ слева.

$A_n = \{n,1,2,\dots, k-1\}$

\begin{alem}[1]
$|F \cap \aset | \leq k$
\end{alem}

\begin{proof}~
\begin{itemize}  

\item 
$\fset \cap \aset = \emptyset \Rightarrow$ очевидно.

\item

$\fset \cap \aset \neq \emptyset \Rightarrow \exists s: A_s \in \fset  $

$\Rightarrow \forall F \in \fset : F \cap A_s \neq \emptyset$

$\Rightarrow \forall A_i \in (\fset \cap \aset) : A_i \cap A_s \neq \emptyset$

\end{itemize}
\end{proof}
 
Пусть $A_s = A_1$.

$
\begin{matrix}
1 & 2 & ~ & k & k+1 & ~ & n \\ 
 * & * & \dots & * & * & \dots & * 
\end{matrix}
$ \quad Здесь первые k элементов $\in A_s$

Перечислим все $A_i \in \aset$, которые пересекаются с $A_s$:

$A_1, A_2, A_3, \dots, A_k; A_n, A_{n-1},\dots, A_{n-k+2}$

Так как $k \leq \frac n 2$, это корректно. Крайние элементы:

$(A_2, A_{n-k+2})$

$(A_3, A_{n-k+3})$

$(A_4, A_{n-k+4})$

\dots

$(A_k, A_{n-k})$

Мы видим, что какое  бы   множество $A_s$ мы ни взяли из $\aset$, мы не сможем взять больше $k-1$ других множеств. Значит, мощность пересечения действительно не больше $k$:

$|F \cap \aset | \leq k$  , что и требовалось доказать.

Применим перестановку $\sigma_i $ к $V$. Поменяются и множества $A_1, \dots, A_n$, т.е. совокупность $\aset \mapsto \aset _{\sigma_i}$. В частности, $\aset_e = \aset$. 

\begin{alem}[2]
$\forall i: |\fset \cap A_{\sigma_i}| \leq k$
\end{alem}
 
Введём функцию-индикатор:

$I(F, A_{\sigma_i} )=
\begin{cases}
1, F \in A_{\sigma_i}\\
0
\end{cases}
$

Вычислим следующую сумму двумя различными способами:

$\sum \limits _{i=1} ^{n!} \sum \limits _{F \in \fset} I(F, A_{\sigma_i} )\leq k n!$ (внутренняя сумма это мощность пересечения из условия леммы)

С другой стороны, фиксируем произвольное $F \in \fset$ и считаем, сколько есть различных перестановок $\sigma_i$, для которых $F \in A_{\sigma_i}$. Найденные количества суммируем по всем $F \in \fset$

Сколько различных перестановок отправляют $F$ в какое-то фиксированное множество?

$\forall i : F \xrightarrow{\sigma} A_i \{1,2, \dots, k\}$

Всего таких $ \sigma $ --- $k! (n-k)!$. Следовательно, существует

$n k! (n-k)!$ перестановок, которые делают из $F$ в $\aset = A_e$.

Обозначим эти перестановки:

$\sigma_{i_1}, \dots, \sigma_{i_{nk!(n-k)!}}$

Возьмём обратные перестановки:

$\sigma_{i_1}^{-1}, \dots, \sigma_{i_{nk!(n-k)!}} ^{-1}$

Тогда:

$\forall \nu = \overline{1,2,\dots, nk!(n-k)!} :F \in A_{\sigma_{i_\nu}^{-1}} $

$\sum \limits _{i=1} ^{n!} \sum \limits _{F \in \fset} I(F, A_{\sigma_i} )= |\fset| nk! (n-k)! $

$|\fset| \leq \frac {kn!} {nk!(n-k)!} = C_{n-1}^{k-1}$
 
\end{enumerate}

\end{proof}
