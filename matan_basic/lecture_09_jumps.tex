\section{Точки разрыва. Свойства функций, непрерывных на отрезке}

\subsection{Точки разрыва}

TODO

\subsection{Свойства функций, непрерывных на отрезке}

\subsubsection{Теорема о нуле функции (Больцано-Коши)}

Если функция на концах отрезка принимает значения противоположных знаков, то существует точка, в которой она равна нулю.

Пусть $f\in C\bigl([a,b]\bigr),$ и $sign(f(a))\ne sign(f(b)).$ Тогда $\exists c \in (a,b)$ такое, что $f(c) = 0.$

В частности любой многочлен нечётной степени имеет по меньшей мере один нуль.

Доказывается по методу деления отрезков (на какой-то точке значение функции будет 0)

\subsubsection{Теорема Коши о промежуточных значениях непрерывной функции }

Неприрывная на отрезке $ [a,b] $ функция принимает все промежуточные значения.

Пусть дана непрерывная функция на отрезке $f\in C\bigl([a,b]\bigr).$ Пусть также $f(a) \neq f(b),$ и без ограничения общности предположим, что $f(a) = A < B = f(b).$ Тогда для любого $C \in [A,B]$ существует $c\in [a,b]$ такое, что $f(c)=C$.

Доказывается сведением к предыдущей теореме.

\subsubsection{Теорема Вейерштрасса}

Теорема Вейерштрасса в математическом анализе и общей топологии гласит, что функция, непрерывная на компакте, ограничена на нём и достигает своей верхней и нижней грани.

Частный случай для мат. анализа

\begin{itemize}
\item 
1-я теорема Вейерштрасса - об ограниченности непрерывной на отрезке функции.
Доказывается из определения (выбором максимального значения 
из значения на конце и максимума (минимума) на отрезке)

2-я теорема Вейерштрасса - о достижении точной верхней и нижней грани непрерывной на отрезке функции

\end{itemize}
В предположениях теоремы отрезок нельзя заменить на открытый интервал. Например, функция тангенс

$ \tg \left( -\frac{\pi}{2} , \frac{\pi}{2} \right) \to \mathbb{R} $

непрерывна в каждой точке области определения, но не ограничена.
