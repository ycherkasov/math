\section{Практика 2. Метод математической индукции}

Использована книга: Соминский - "Метод математической индукции"

\subsubsection{Пример 1}

Рассмотрим сумму $ n $ нечетных чисел.

$ S_{n} = 1 + 3 + \ldots + (2n-1) $

Заметим, что $ S_{1} = 1, S_{2} = 4, S_{3} = 9, S_{4} = 16, \ldots $, т.е., квадрат числа $ n $.

Докажем по индукции:

$ S_{n} = n^{2} $

$ S_{n+1} = (n)^{2} + (2n + 1) = (n+1)^{2}$

Доказано

\subsubsection{Пример 2}

TODO