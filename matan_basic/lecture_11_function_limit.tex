\section{Семинар: Предел функции}

\subsection{Замечательные пределы}

\subsubsection{Первый замечательный предел}

$\lim\limits_{x \to 0}\dfrac{\sin x}{x} = 1$


Следствия:

\begin{itemize}
\item 
$\lim\limits_{x \to 0}\dfrac{\mathrm{tg} x}{x} = 1$

\item 
$\lim\limits_{x \to 0}\dfrac{\arcsin x}{x} = 1$

\item 
$\lim\limits_{x \to 0}\dfrac{\mathrm{arctg} x}{x} = 1$

\item 
$\lim\limits_{x \to 0}\dfrac{1 - \cos x}{ \frac{x^2}{2} } = 1$
\end{itemize}

\subsubsection{Второй замечательный предел}

$\lim\limits_{x \to \infty}\left(1 + \frac{1}{x}\right)^x = e$

Следствия:

\begin{itemize}
\item 
$\lim\limits_{u \to 0}(1 + u)^\frac{1}{u}=e$

\item 
$\lim\limits_{x \to \infty}\left(1 + \frac{k}{x}\right)^x = e^k$

\item 
$\lim\limits_{x \to 0}\dfrac{\ln(1 + x)}{x} = 1$

\item 
$\lim\limits_{x \to 0}\dfrac{e^x - 1}{x} = 1$

\item 
$\lim\limits_{x \to 0}\dfrac{a^x - 1}{x \ln a} = 1$ для $a > 0 \,\!$, $a \neq 1 \,\!$

\item 
$\lim\limits_{x \to 0}\dfrac{(1 + x)^\alpha - 1}{\alpha x} = 1$

\end{itemize}

\subsection{Паттерны тривиальных пределов}

\begin{enumerate}
\item 
Присутствуют радикалы - домножаем на сопряженное
$\lim\limits_{x\to x_0} \sqrt{a} - \sqrt{b} = 
\frac{(\sqrt{a} - \sqrt{b})(\sqrt{a} + \sqrt{b})}{(\sqrt{a} + \sqrt{b})} =
\frac{a - b}{(\sqrt{a} + \sqrt{b})}  = \cdots $

\item 
Сложная функция - используем предельный переход в сложных функциях
$ \lim\limits_{x\to x_0} \log_a{f(x)} =  \log_a {\lim_{x\to x_0} {f(x)} } =  $ \\
$ \lim\limits_{x\to x_0} e^{f(x)} =  e^ {\lim_{x\to x_0} {f(x)} }  $

\end{enumerate}


\subsection{Пример 1. Замена переменной}

$ \lim\limits_{x\to 0} \dfrac{ \sqrt[5]{32+x} - 2 }{x} $

Замена переменных:

$ y = \sqrt[5]{32+x} - 2 $

$ y^{5} = 32+x $

$ x = y^{5} - 32 $

$ x \to 0, y \to 2 $

В результате замены:

$ \lim\limits_{x\to 0} \dfrac{ \sqrt[5]{32+x} - 2 }{x} = $
$ \lim\limits_{y\to 2} \dfrac{ y - 2 }{ y^{5} - 32 } =  $
$ \lim\limits_{y\to 2} \dfrac{ y - 2 }{ (y - 2)(y^{4}-2y^{3}+4y^{2}+8+16) } = \dfrac{1}{80}  $

\subsection{Паттерны сведения к замечательным пределам}

\subsubsection{Пример 2. Сведение к первому замечательному пределу}

Если есть синус, все просто
$ \lim\limits_{x\to 0} \dfrac{ \sin{\alpha x} }{x} = \alpha $

\subsubsection{Пример 3. Сведение к первому замечательному пределу}

Любые тригонометрические выражения нужно стараться свести к
1 ЗП с помощью тригонометрических формул.

$ \lim\limits_{x\to 0} \dfrac{ \cos{3x} - \cos{7x}}{x^{2}} = $
$ \lim\limits_{x\to 0} -2 \dfrac{\sin{5x}}{x} \dfrac{\sin{2x}}{x} = -20 $
(по формуле разности косинусов)

\subsubsection{Пример 3. Замена переменных в тригонометрических выражениях}

$ \lim\limits_{x\to \frac{\pi}{4} } \mathrm{ctg}(2x)  \mathrm{ctg}( \frac{\pi}{4} - x)  $

Замена переменных:

$ x = \frac{\pi}{4} $

$ y \to 0 $

$ x = \frac{\pi}{4} - y $

$ 2x = \frac{\pi}{2} - 2y $

В результате замены:

$ \lim\limits_{x\to \frac{\pi}{4} } \ctg{2x}   \ctg( \frac{\pi}{4} - x) = $
$ \lim\limits_{x\to \frac{\pi}{4} } \ctg(\frac{\pi}{2} - 2y)  \ctg{y} = $
$ \lim\limits_{x\to \frac{\pi}{4} } 
\dfrac{\sin(\frac{\pi}{2} - 2y)}{\cos(\frac{\pi}{2} - 2y)}
\dfrac{\cos{y} }{\sin{y} } = $

По формуле сдвига на $ \dfrac{\pi}{2} $ 

$ = \dfrac{\sin{2y} }{\cos{2y} } \dfrac{\cos{y} }{\sin{y} } = $
$ \dfrac{\sin{2y} }{\sin{y} } = 2 $
