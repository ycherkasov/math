\section{Непрерывность функции. Основные элементарные функции. Замечательные пределы. Операции над непрерывными функциями}

\subsection{Определения}

Непрерывная функция — функция без «скачков», то есть такая, у которой сколь угодно малые изменения аргумента приводят к сколь угодно малым изменениям значения функции.
\\\\
Определение непрерывности фактически повторяет определение предела функции в данной точке. Другими словами, функция $f$ непрерывна в точке $x_0$, предельной для множества $E$, если $f$ имеет предел в точке $x_0$, и этот предел совпадает со значением функции $f(x_0)$.
\\\\
Функция непрерывна в точке, если её колебание в данной точке равно нулю.

\subsection{Теоремы о непрерывных функциях}

\begin{enumerate}

\item 
Теорема о сохранении неравенства

Если $ f(x) $ непрерывна в (.) $ x_0;  f(x) > A $, то $ \exists \delta $ такое, что 
неравенство сохраняется в интервале $ (x_0 - \delta , x_0 + \delta) $

Доказывается по определению предела. Берем $ \epsilon = \frac{h}{2} $, подставляем значения в неравенство в определении, и получаем в левой части значение $ A + \frac{h}{2} > A $

\item 
Теорема об устойчивости знака.

Так же, как и предыдущая, только $ A = 0 $

\end{enumerate}

Любая комбинация непрерывной функции является непрерывной (кроме деления на бмф)

\subsection{Сложная функция}

Сложная функция (суперпозиция функций) — это применение одной функции к результату другой.

Предельный переход под знаком непрерывной функции

$ \lim f( \phi(x) ) = f( \lim  \phi(x) ) $

\subsubsection{Пример использования перехода}

$ \lim\limits_{x \to 0} \dfrac{ \ln(1+x) }{x} $
$ = \lim\limits_{x \to 0}  \ln(1+x)^{\frac{1}{x}} $
$ = \ln (\lim\limits_{x \to 0}  (1+x)^{\frac{1}{x}}) = \ln e = 1 $

Применяется свойство логарифма с внесением коэффициента в показатель степени.

\subsubsection{Теорема о непрерывности сложной функции}

Доказывается просто из определения предела и сложной функции.