\section{Семинар: Предел функции. Замечательные пределы}

\subsection{Обобщение 2го ЗП}

При решении многих пределов полезно использовать соледующий паттерн:

Если $ \exists  \alpha(x), \beta(x) $ такие, что 

$ \lim\limits_{x \to 0 } \alpha(x) = 0 $

$ \lim\limits_{x \to 0 } \beta(x) = 0 $

$ \lim\limits_{x \to a } \dfrac{\alpha(x)}{\beta(x)} = A $

$$
\lim\limits_{x \to a } \left( 1 + \alpha(x)  \right )^{\frac{1}{\beta(x)}} = e^{A}
$$

\subsection{Пример 1}

$ \lim\limits_{x \to 0} (1 - \tg^{2}x)^{1/\sin^{2}2x}  = $
$ \lim\limits_{x \to 0} (1 + ( - \tg^{2}x ) )^{1/\sin^{2}2x}$

Проверим выполнимость условий

$ \lim\limits_{x \to 0} \tg^{2}x = \lim\limits_{x \to 0} \sin^{2}2x  = 0$

$ \lim\limits_{x \to 0} \dfrac{- \tg^{2}x}{ \sin^{2}2x } = $
$ \dfrac{- x^{2}}{ 4x^{2} } = - \dfrac{1}{4}$
(по эквивалентностям БМФ)



$ \lim\limits_{x \to 0} (1 - \tg^{2}x)^{1/\sin^{2}2x}  = e^{-\frac{1}{4}}$

\subsection{Пример 2}

Выражения со сложным показателем степени нужно стараться приводить к виду
$ \lim\limits_{x \to a } \left( 1 + \alpha(x)  \right )^{\frac{1}{\beta(x)}} $

Например

$ \lim\limits_{x \to 0} (\cos x)^{\frac{1}{x^{2}}} = $
$ \lim\limits_{x \to 0} (1 + (1- \cos x))^{\frac{1}{x^{2}}} =$
$ \lim\limits_{x \to 0} ((1 - 2 \sin^{2} \dfrac{x}{2} ))^{\frac{1}{x^{2}}} = $

Проверим выполнимость условий

$ \lim\limits_{x \to 0} \sin^{2} \dfrac{x}{2} = \lim\limits_{x \to 0} x^{2} = 0 $

$ \lim\limits_{x \to 0} \dfrac{ 2 \sin^{2} \dfrac{x}{2} }{ x^{2}} =  $
$ \lim\limits_{x \to 0} \dfrac{ 2 \dfrac{x^{2}}{4} }{ x^{2}} = -\dfrac{1}{2} $

$ \lim\limits_{x \to 0} (\cos x)^{\frac{1}{x^{2}}} = e^{-\frac{1}{2}}$

\subsection{Пример 3}

Дроби с показателем степени удобно разбивать на два вырежения вида
$ \lim\limits_{x \to a } \left( 1 + \alpha(x)  \right )^{\frac{1}{\beta(x)}} $


Например

$ \lim\limits_{x \to \infty } \left( \dfrac{ 2x^{2}+3 }{ 2x^{2}-1 }  \right)^{x^{2}} = $

Введем замену переменных $ t = \dfrac{1}{x^{2}} $

$ \lim\limits_{x \to \infty } \left( 
\dfrac{ 1+\frac{3}{2x^{2}} }{ 1- \frac{1}{2x^{2}}  }  
\right)^{x^{2}} = $
$ \lim\limits_{t \to 0 } \left( 
\dfrac{ 1 + 3/2t }{ 1 - 1/2t }  
\right)^{\frac{1}{t}} = $
$ 
\dfrac
{ \lim\limits_{t \to 0 } (1 + 3/2t)^{\frac{1}{t}} }
{ \lim\limits_{t \to 0 } (1 - 1/2t)^{\frac{1}{t}} } = $

Вычислим по отдельности

$ \lim\limits_{t \to 0 } \dfrac{3/2 t}{t} = 3/2 $

$ \lim\limits_{t \to 0 } - \dfrac{1/2 t}{t} = -1/2 $

$ \lim\limits_{x \to \infty } \left( \dfrac{ 2x^{2}+3 }{ 2x^{2}-1 }  \right)^{x^{2}} 
= e^{\frac{3}{2}-(-\frac{1}{2})} = e^{2}$

\subsection{Пример 4. Пределы на $ +\infty /-\infty$}

На $ +\infty $ считается так же, как и на $ \infty $.

На $ -\infty $ делаем замену переменных $ x = -t $ и дальше считаем как на  $ +\infty $.

\subsubsection{На $ +\infty $}

$ \lim\limits_{x \to +\infty } \dfrac{ \sqrt{x^{2}+14} + x }{ \sqrt{x^{2}-2} + x } = $
$ \lim\limits_{x \to +\infty } \dfrac{ \sqrt{1+14/x^{2}} + 1 }{ \sqrt{1-2/x^{2}} + 1 } = 1 $ \\\\


\subsubsection{И на $ -\infty $}


$ \lim\limits_{x \to -\infty } \dfrac{ \sqrt{x^{2}+14} + x }{ \sqrt{x^{2}-2} + x } $

Введем замену переменных $ t = -x $

$ \lim\limits_{x \to -\infty } \dfrac{ \sqrt{x^{2}+14} + x }{ \sqrt{x^{2}-2} + x } = $
$ \lim\limits_{t \to +\infty } \dfrac{ \sqrt{t^{2}+14} - t }{ \sqrt{t^{2}-2} - t } $

Решается домножением на сопряженные или по правилу Лопиталя.

$ \lim\limits_{t \to +\infty } \dfrac{ \sqrt{t^{2}+14} - t }{ \sqrt{t^{2}-2} - t } = $
$ \lim\limits_{t \to +\infty }
\dfrac
{ (\sqrt{t^{2}+14} - t)(\sqrt{t^{2}+14} + t)(\sqrt{t^{2}-2} - t) }
{ (\sqrt{t^{2}-2} - t)(\sqrt{t^{2}+14} + t)(\sqrt{t^{2}-2} - t) } = -7$
