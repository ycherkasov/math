\section{Практика 1. Метод математической индукции}

\subsection{Демидович}

\subsubsection{Пример 1}

$ 1 + 2 + \ldots + n = \dfrac{n(n+1)}{2} $

Для 1,2 - тривиально

Для $ k+1 $

$ 1 + 2 + \ldots + k + (k+1) = $
$ \dfrac{k(k+1)}{2} + (k+1) =  $
$ \dfrac{k(k+1) + 2(k+1)}{2} = $
$\dfrac{(k+1)(k+2)}{2} =$
$\dfrac{(k+1)((k+1) + 1)}{2}$

Доказано

\subsubsection{Пример 2}

См. семинар

\subsubsection{Пример 3}

$ 1^{3} + 2^{3} + \ldots + n^{3} = (1 + 2 + \ldots + n)^{2} $

Для 1,2 - тривиально

Для $ k+1 $ преобразуем обе части тождества
$ 1^{3} + 2^{3} + \ldots + k^{3} + (k+1)^{3} = (1 + 2 + \ldots + k + (k+1))^{2} $

\begin{enumerate}

\item 
$ 1^{3} + 2^{3} + \ldots + k^{3} + (k+1)^{3} =$
$ (1 + 2 + \ldots + k)^{2} + (k+1)^{3} = $
$ \left( \dfrac{k(k+1)}{2} \right)^{2} + (k+1)^{3}  $

\item 
$ (1 + 2 + \ldots + k + (k+1))^{2} = $
$   \left( \dfrac{k(k+1)}{2} + (k+1) \right)^{2} = $

$ = \left( \dfrac{k(k+1)}{2} \right)^{2} + \dfrac{2k(k+1)(k+1)}{2} + (k+1)^{2} = $

$  = \left( \dfrac{k(k+1)}{2} \right)^{2} + k(k+1)^{2} + (k+1)^{2} =  $

$  = \left( \dfrac{k(k+1)}{2} \right)^{2} + (k+1)(k+1)^{2} =  $
$  \left( \dfrac{k(k+1)}{2} \right)^{2} + (k+1)^{3} $
\end{enumerate}

Обе части тождества равны $  \left( \dfrac{k(k+1)}{2} \right)^{2} + (k+1)^{3} $

Доказано

\subsubsection{Пример 4}

$ 1 + 2 + 2^{2} + \ldots + 2^{n-1} = 2^{n} - 1  $

Для 1,2 - тривиально

Для $ k+1 $ преобразуем левую часть тождества
$ 1 + 2 + 2^{2} + \ldots + 2^{k-1} + 2^{k} = 2^{k+1} - 1  $

$ 2^{k+1} - 1 = (2^{k} - 1) + 2^{k} = 2 \cdot 2^{k} - 1 = 2^{k+1} - 1 $

Доказано

\subsubsection{Пример 5}

TODO

\subsubsection{Примеры 6, 7}

Разновидности неравенства Бернулли (см. семинар)

$(1+x^{1})(1+x^{2}) \ldots (1+x^{n}) \ge 1 + x^{1} + x^{2} + \ldots + x^{n}$
$; (x \ge -1, n \in N  ) $

Для $ n = 1,2,3 $ - очевидно

Предположим верно для $k$:
$(1+x^{1})(1+x^{2}) \ldots (1+x^{k}) \ge 1 + x^{1} + x^{2} + \ldots + x^{k}$

Прибавим к каждой части неравенства положительное число $ x^{k+1} $

$ \underbrace{(1+x^{1})(1+x^{2}) \ldots (1+x^{k})}_{\ge 1 + x^{1} + x^{2} + \ldots + x^{k}} $
$ + x^{k+1} \ge 1 + x^{1} + x^{2} + \ldots + x^{k} + x^{k+1}$

Т.к. часть неравенства над скобкой и так больше или равна 
$ 1 + x^{1} + x^{2} + \ldots + x^{k} $,
добавление положительного числа этого неравенства не изменит

\subsubsection{Пример 8}

$ n! >  \left( \dfrac{n+1}{2} \right)^{n}, n > 1 $

Для $ n = 2,3 $ - очевидно

Для $ k $: 

$ k! >  \left( \dfrac{k+1}{2} \right)^{k} $

Домножим обе части неравенства на $ k $

$ (k+1)! >  k \left( \dfrac{k+1}{2} \right)^{k} $

Докажем, что это неравенство сильнее, чем

$ (k+1)! >  \left( \dfrac{k+1}{2} \right)^{k+1} $

$ k \left( \dfrac{k+1}{2} \right)^{k} >  \left( \dfrac{k+1}{2} \right)^{k+1} $

$ k \left( \dfrac{k+1}{2} \right)^{k} >  \left( \dfrac{k+1}{2} \right)^{k} \left( \dfrac{k+1}{2} \right) $

$ k > \dfrac{k+1}{2} $

$ k > \dfrac{1+2+ \ldots +k}{k} $

В левой части неравенства все $ k $ слагаемых, кроме одного, меньше $ 1 $

В правой - $ k $ единиц.

Доказано.